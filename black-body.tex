Будем исходить из формулы \eqref{eq:mods}. Спектральная плотность равна
\[
  u_{\omega, T}\,d\omega = \frac{dN\cdot\langle \varepsilon \rangle}{V} =
  \frac{\omega^2}{\pi^2 c^3}\langle \varepsilon \rangle,
\]
где $ \langle \varepsilon \rangle $ --- средняя энергия волны частоты $ \omega
$. Согласно классической теореме о равномерном распределении энергии по степеням
свободы, в состоянии термодинамического равновесия на каждую степень свободы
системы\footnote{Их в данном случае две: магнитная и электрическая.} приходится
в среднем энергия $ kT/2 $, где $ k $ --- постоянная Планка. Выше была получена
формула, позволяющая найти испускательную способность АЧТ в виде 
\[
  r^\ast_{\omega, T} = \frac{\omega^2}{4\pi^2 c^2}kT.
\]
Эта формула носит имя \emph{формулы Рэлея -- Джинса}. Опыты резко расходятся с ней
при больших длинах волн, однако не нужен опыт, чтобы заметить её абсурдность ---
следуя этой формуле, $ u(T) = 4R^\ast/c = \infty $.

В предположении дискретности энергии излучения $ E = \hbar \omega $ получим иную
формулу 
\[
  \langle \varepsilon \rangle = \sum_{n=0}^\infty P_n\varepsilon_n,
\]
где $ \varepsilon = n\hbar\omega $ --- возможные значения энергии; $ P_n $ ---
вероятность его получения в термодинамическом равновесии при данной температуре.
Определим её с помощью распределения Больцмана: 
\[
    P_n = A\exp \left( - \frac{\varepsilon_n}{kT} \right),
\]
где $ A $ нормируется соответствующим образом. Тогда 
\[
  \langle\varepsilon\rangle = \hbar \omega \frac{\sum n \exp(-n\xi)}{\sum
  \exp(-n\xi)},
\]
где $ \xi = (\hbar\omega)/(kT) $.
Поскольку по формуле геометрической прогрессии 
\begin{align*}
  S &= \sum e^{-n\xi} = \frac{1}{1 - e^{-\xi}},\\
  -\frac{dS}{d\xi} &= \sum ne^{-n\xi}=\frac{e^{-\xi}}{(1-e^{-\xi})^2},
\end{align*}
имеем следующую \emph{формулу Планка} 
\[
  r^\ast_{\omega, T} =
\frac{\hbar\omega^3}{4\pi^2c^2(\exp\left(\frac{\hbar\omega}{kT}\right)-1)}.
\]

