\documentclass[12pt]{article}

\usepackage[T2A]{fontenc}
\usepackage[utf8]{inputenc}
\usepackage[english,russian]{babel}

\usepackage{amsmath}
\usepackage{amsthm, mathrsfs, mathtools, amssymb}
\usepackage{enumitem}

\usepackage{epigraph}

\usepackage{tikz}
\usetikzlibrary{decorations.pathmorphing}
% для волнистой линии для фотонов создал стиль линии snake arrow
% рисует волну и на конце ее чуть-чуть прямую линию оставляет для стрелки
\tikzset{snake arrow/.style=
	{->,
		decorate,
		decoration={snake,amplitude=.4mm,segment length=2mm,post length=1mm}},
}

\usepackage{float}
\usepackage{multicol}
%\usepackage{hyperref}
%\hypersetup{
%	colorlinks,
%	citecolor=black,
%	filecolor=black,
%	linkcolor=black,
%	urlcolor=black
%}

\usepackage{geometry}
\geometry{verbose,a4paper,tmargin=1cm,bmargin=2cm,lmargin=1.5cm,rmargin=1.5cm}

\begin{document}
%  \tableofcontents \newpage
  
  \section{Письменная часть}
  
  \subsection{Спонтанное и вынужденное излучение. Вывести соотношение Эйнштейна для
коэффициентов A и B}\label{einstein-a-b}
%ref: M, 300
Пусть в полости с веществом, состоящем из одинаковых не взаимодействующих друг с
другом частиц, наблюдается термодинамическое равновесие. Для некоторой частицы с дискретным спектром энергий рассматриваем поведение при налетающем
(индуцирующим, в том плане, что оно вызывает вынужденное излучение) фотоне. Эта частица может
перейти на уровень выше, поглотив фотон, может перейти на уровень ниже, испустив
ещё один фотон вдобавок.
В случае испускания дополнительного фотона данное явление называется
\emph{вынужденным излучением}.

Также частица может самопроизвольно выпустить фотон, потеряв часть энергии и опустившись на
уровень ниже. Это явление называется \emph{спонтанным излучение}.

Обозначим $E_1, E_2$ --- энергии основного и возбужденного состояний\footnote{В
этой упрощённой модели, считаем, что больше уровней энергии нет.}, $N_1, N_2$ --- количество
частиц в основном и возбужденном состоянии. Тогда за некоторый промежуток $\Delta t$ произойдёт
какое-то количество переходов $\Delta n_{12}$ из состояния 1 в 2 и $\Delta n_{21}$ из
состояния 2 в 1.

Согласно распределению Ф-Д (Больцмана)
\[
  N_i \sim e^{-\dfrac{E_i}{kT}},\quad \frac{N_2}{N_1} = e^{-\dfrac{E_2 - E_1}{kT}},
\]
где $ k $ --- постоянная Больцмана.

Заметим что при данных условиях, излучение в полости монохроматическое с
частотой $ \omega $ такой, что $ E_2 - E_1 = \hbar\omega $. Для падающего излучения вспомним такую характеристику, как спектральная плотность излучения:
\[
  u_\omega = \dfrac{dW}{dV d\omega}.
\]
Она характеризует, если утрировать, количество фотонов в некотором элементе объёма.

\subsubsection{Случай отсутствия вынужденного излучения}
Предположим, что при налетающем фотоне может произойти только либо поднятие $1 \to 2$, 
либо просто как-то спонтанно произойти излучение $2 \to 1$. То есть предположим, что нет никакого
вынужденного излучения и это всё ложь.

Обозначим $z_{12}$ --- количество частиц, которые за единицу времени перейдут из 1 во
2-ое состояние, $z_{21}$ -- из 2 в 1. Ясно, что каждая из этих величин пропорциональна количеству
частиц на данном энергетическом уровне. Так же ясно, что частота переходов $1\to 2$, вызываемых
поглощением фотона, пропорциональна количеству налетаемых фотонов. Тогда:
\[
  \begin{cases}
    z_{12} = \dfrac{\Delta n_{12}}{\Delta t} \sim N_1 \cdot u_\omega, \\[10pt]
    z_{21} = \dfrac{\Delta n_{21}}{\Delta t} \sim N_2. 
  \end{cases}
  \Rightarrow
  \begin{cases}
    z_{12} = B_{12} N_1 u_\omega, \\
    z_{21} = A_{21} N_2.
  \end{cases}
\]
здесь коэффициенты $B_{12}$ и $A_{21}$ появились просто как константы
пропорциональности (вероятности переходов).
В термодинамическом равновесии за единицу времени в обе стороны переходит одинаковое количество 
частиц (принцип детального равновесия):
\begin{equation}\label{write_01:bad_einstein}
  z_{12} = z_{21} \Rightarrow
  B_{12} N_1 u_\omega = A_{21} N_2
  \Rightarrow
  B_{12} u_\omega = A_{21} \dfrac{N_2}{N_1}.
\end{equation}
Рассмотрим что будет происходить при нагреве $T \to \infty$, согласно распределению Ф-Д:
\[
  \dfrac{N_2}{N_1} = e^{- \dfrac{E_2-E_1}{kT}} \to 1,
\]
Из уравнения \eqref{write_01:bad_einstein} получаем, что:
\[
  \dfrac{A_{21}}{B_{12}}
  \approx u_\omega
  = \dfrac{\hbar \omega^3}{\pi^2 c^3} \cdot \dfrac{1}{e^{\dfrac{\hbar \omega}{kT}} - 1} \to \infty.
\]
(здесь использован закон Планка) получаем противорение -- не существует таких коэффициентов, чтобы
такая модель (без вынужденного излучения) работала.

\subsubsection{Случай вынужденного излучения}
Чтобы модель стала верной, надо добавить вынужденное излучнение. В этом случае к $z_{21}$ добавяться
ещё и частицы, которые выпустят фотон вследствие вынуждения:
\[
  \begin{cases}
    z_{12} = B_{12} N_1 u_\omega, \\
    z_{21} = A_{21} N_2 + B_{21} N_2 u_\omega.
  \end{cases}
\]
в термодинамическом равновесии за единицу времени в обе стороны переходит одинаковое количество 
частиц:
\begin{equation}\label{write_1:good_einstein}
  z_{12} = z_{21} \Rightarrow
  B_{12} N_1 u_\omega = A_{21} N_2 + B_{21} N_2 u_\omega
  \Rightarrow
  B_{12} \dfrac{N_1}{N_2} = \dfrac{A_{21}}{u_\omega} + B_{21}.
\end{equation}
При $T \to \infty$:
\[
  \dfrac{N_1}{N_2} = e^{- \dfrac{E_2-R_1}{kT}} \to 1,
  \dfrac{1}{u_\omega} = \dfrac{\pi^2 c^3}{\hbar \omega^3} \left(e^{\dfrac{\hbar \omega}{kT}} - 1\right) \to 0
  \Rightarrow
  \begin{cases}
    B_{12} = B_{21} = B, \\
    A_{21} = A 
  \end{cases}
\]
Таким образом:
\begin{multline*}
  B N_1 u_\omega = A N_2 + B N_2 u_\omega
  \Rightarrow
  B u_\omega \left( \dfrac{N_1-N_2}{N_2} \right) = A
  \Leftrightarrow \\
  \Leftrightarrow
  B u_\omega \left( \dfrac{N_1}{N_2} - 1 \right) = B u_\omega \left( e^{\dfrac{E_2-E_1}{kT}} - 1 \right) = A
  \Leftrightarrow
  B \dfrac{\hbar \omega^3}{\pi^2 c^3} = A
\end{multline*}
Последнее называется соотношением Эйнштейна.
 \newpage
  \subsection{Сформулировать квантовомеханическую задачу для гармонического осциллятора. Получить
его решение. Указать, на каком этапе решения возникает квантование энергии. Правила отбора
для гармонического осциллятора.}
Рассмотрим гармонические колебания в одномерном случае. В классической физике
имеем\footnote{Понятно, что в реальности такой силы сущестовать не может и
рассмотрение ограничивается приближёнными малыми колебаниями.} $ F_x = -kx $, и, значит,  
\begin{equation}
    U(x) = \frac{kx^2}{2} = \frac{m_0\omega_0^2 x^2}{2},
    \label{eq:potenc}
\end{equation}
где $ \omega_0^2 = k/m_0 $ --- квадрат собственной частоты классического
гармонического осцилятора. Таким образом, рассматривается движение частицы в
параболической потенциальной яме. 
\begin{figure}[h]
  \centering
  \includegraphics[width=0.3\textwidth]{img/write-02/yama.png}
  \caption{Потенциальная энергия гармонического осциллятора.}
  \label{fig:yama}
\end{figure}
На рисунке \ref{fig:yama} число $ a_0 $ есть, конечно,
амплитуда колебаний. Считая без потери общности движение равным $ x = a_0 \cos\omega t $, легко
получим 
\[
    E = \frac{m_0a_0^2\omega_0^2}{2}, \quad a_0 =
    \sqrt{\frac{2E}{m_0\omega_0^2}}.
\]

Перейдём к квантовому случаю. А именно, решим уравнение Шрёдингера
для потенциальной энергии \eqref{eq:potenc}: 
\[
    \frac{\hbar^2}{2m} \psi''(x) + \left(E - \frac{kx^2}{2}\right) \psi(x) = 0.
\]

\subsubsection{Решение уравнения}
\textbf{1. Масштабирование}. Перейдём к безразмерным величинам $ \varepsilon =
E/E_0 $, $ \xi = x/x_0 $. Здесь $ E_0 = E_{\min} = \hbar \omega/2 = kx_0/2 $.
Обоснуем: 
\begin{align*}
  \Delta x &= \sqrt{\langle x^2\rangle} = \sqrt{\frac{1}{2} a_0^2} =
  \sqrt{\frac{E}{m_0\omega_0^2}},\\
  \Delta p &= \sqrt{\langle p^2 \rangle} =
  \sqrt{\frac{1}{2}m_0^2a_0^2\omega_0^2} = \sqrt{m_0E},
\end{align*}
откуда вместе с неравенством Гейзенберга $ \Delta x \Delta p \geqslant \hbar/2 $
и следует результат. Итак, перейдём к величинам $ \varepsilon $, $ \xi $ и
получим уравнение
\begin{equation}
  \psi'' + (\varepsilon - \xi^2)\psi = 0.  
  \label{eq:eq1}
\end{equation}

\textbf{2. Асимптотика.} Устремим в уравнении $ |\xi| \to \infty $. Тогда
главная часть примет вид 
\[
    \psi'' = \xi^2 \psi.
\]
Хорошо бы $ \psi \to 0 $, откуда вытекает предположение о виде волновой функции
$ \psi = e^{\alpha(\xi)} $, где $ \alpha(\xi) \to -\infty $. Тогда $ \psi'' =
(\alpha'' + (\alpha')^2)e^\alpha $, а значит, $ \alpha'' + (\alpha') = \xi^2 $.
Тогда (?) $ \alpha = A\xi^n $. В этом предположении в связи с рассматриваемой
асимптотикой будем рассматривать только слагаемое в квадрате, получим $
A^2n^2\xi^{2n-2} \approx \xi^2 $, откуда $ A = -1/2 $, $ n= 2 $. Искомая
асимптотика --- $ \psi  \to \exp( - \xi^2/2) $.

\textbf{3. Общий вид.} Будем в таком случае искать решение в виде $ \psi(\xi) =
\exp(-\xi^2/2)f(\xi)$. Подставив в \eqref{eq:eq1} и упростив выражение, получим 
\begin{gather*}
    \psi'' = \exp(-\xi^2/2)^{(2)}f + 2\exp(-\xi^2/2)^{(1)}f^{(1)} +
    \exp(-\xi^2/2) f^{(2)} = \exp(-\xi^2/2)(f''-2\xi f' + (\xi^2 - 1)f),\\
    f'' - 2\xi f' + (\varepsilon - 1)f = 0.
\end{gather*}
Саму функцию $ f $ представим рядом $ f = \sum\limits_{k=0}^{+\infty} a_k
\xi^k$. Тогда  
\begin{align*}
  f' &= \sum_{k=0}^\infty ka_k \xi^{k-1},\\
  f'' &= \sum_{k=0}^\infty k(k-1) a_k
  \xi^{k-2} = \sum_{k=0}^\infty (k+1)(k+2) a_{k+2}\xi^k.
\end{align*}
Получаем для любого $ \xi $  
\[
  \sum_{k=0}^\infty ((k+2)(k+1)a_{k+2} + (-2k + \varepsilon - 1) a_k)\xi^k = 0.
\]
Благодаря произвольности $ \xi $ можем вывести теперь рекуррентное соотношение 
\begin{equation}
  a_{k+2} = \frac{2k+1-\varepsilon}{(k+2)(k+1)}a_k \approx \frac{2}{k}a_k
  \label{eq:rek}
\end{equation}
при $ k \to \infty $. С другой стороны для некоторого $ k' = k/2 $
\[
  \exp(\xi^2) = \sum_{k'=0}^\infty \frac{\xi^{2k'}}{k'!} = \sum_{k=0}^\infty
  \frac{\xi^k}{(k/2)!},
\]
где для последнего ряда также выполняется предельное рекуррентное соотношение
\eqref{eq:rek}. Поэтому и примем $ f(\xi) = \exp(\xi^2) $. Однако в этом случае
$ \psi = \exp(\xi^2/2) \to \infty $ при $ \xi \to \infty $.

\textbf{4. Квантование.} Приходим к тому, что ряд $ f(\xi) $ нужно оборвать.
Рассмотрим индекс последнего ненулевого члена $ n := k_{\max} $ и нулевой член $ a_{n+2} $. Из реккурентного
соотношения \eqref{eq:rek} вытекает теперь, что $ \varepsilon = 2n+1 $. Заметим,
что этот факт и говорит о квантовании энергии. Сама энергия имеет вид
\[
  E_n =
  \frac{\hbar \omega_0}{2} + n\hbar\omega_0.
\]

\textbf{5. Полиномы Чебышёва -- Эрмита.} Получили, что $ f(\xi) $ есть некие
полиномы степени $ n $. Конкретный вид полиномов выберем из соображений
ортогональности решений. Ими будут полиномы Эрмита  
\[
  H_n := (-1)^n \exp(x^2) \frac{d^n}{d\xi^n}\exp(-x^2),
\]
которые как раз имеют
\emph{вес} $ \exp(-\xi^2) $, то есть 
\[
  \int\limits_{-\infty}^{\infty}H_n H_m \exp(-\xi^2)\,d\xi =
  \frac{\delta_{nm}}{A_nA_m},
\]
где коэффициент 
\[
  A_k = \frac{1}{\sqrt{2^kk!\sqrt{\pi}}}.
\]

Таким образом, волновые функции будут равны 
\[
    \psi_n(\xi) = A_n \exp(-x^2/2) H_n(\xi).
\]
\qed

\begin{figure}[h]
  \centering
  \includegraphics[width=0.8\textwidth]{img/write-02/psi.png}
  \caption{Волновые функции гармонического осциллятора}
  \label{fig:psi}
\end{figure}

Обратим внимание, что энергия, а значит, амплитуда $ a_0 $ зависят от $ n $. При
этом в отличие от прямоугольных ям энергия с переходом на новый уровень меняется
на равные отрезки $ \Delta E = \hbar\omega_0 $. Точный расчет
показывает, что особенности испускания и поглощения 
электромагнитного излучения гармоническим осциллятором таковы, что 
возможны переходы только между соседними уровнями. Условия, которые определяют
изменение квантовых чисел при разрешенных переходах системы из одного состояния в другое,
называются правилами отбора. Таким образом, согласно правилам
отбора, квантовое число $ n $ при испускании и поглощении 
электромагнитного излучения квантовым гармоническим осциллятором
может изменяться только на единицу.

Снова минимальная энергия положительна, что играет очень важную роль в физике (см., напр.,
отсутствие кристаллизация гелия при абсолютном нуле и нормальном давлении).

Попробуем рассчитать вероятность для классического осциллятора. Там $ x =
a_0\sin \frac{2\pi}{T}t $, $ dx = a_0 \frac{2\pi}{T} \cos \frac{2\pi}{T}t\,dt $,
и вероятность $ dP $ того, что
частица при движении в одну сторону находится в интервале шириной $ dx $,
равна 
\[
  dP = \frac{dt}{T/2} = \frac{dx}{\pi\sqrt{a_0^2 - x^2}}.
\]
При больших $ n $ функция $ |\psi_n|^2 $ похожа на $ dP/dx $, однако для малых
$ n $ они отличаются очень сильно.

\begin{figure}[h]
  \centering
  \includegraphics[width=0.8\textwidth]{img/write-02/dPdx.png}
  \caption{Плотности вероятности обнаружения частицы для 
квантового (сплошная линия) и классического (пунктирная линия) осциллятора.}
  \label{fig:dPdx}
\end{figure}
 \newpage
  \subsection{Записать операторные уравнения для проекции момента импульса $L_z$ и для квадрата момента
импульса $L^2$. Получить решение уравнения для $L^2$. Показать, что из условия однозначности
и конечности $\psi$-функции следует квантование величин $L_z$ и $L^2$ }


 \newpage
  \subsection{Записать уравнение Шрёдингера для атома водорода в сферических координатах и произвести
разделение переменных. Решить уравнение для радиальной части $R(r)$ волновой функции. Показать,
что из решения уравнения для $R(r)$ при выполнении условий регулярности следует дискретность
энергии атома водорода.}


 \newpage
  \subsection{Сформулировать, какие частицы называются фермионами, привести примеры Ферми-частиц.
Сформулировать принцип Паули (W. Pauli). Используя формулу Больцмана для энтропии, вывести
распределение Ферми-Дирака} \label{fermi-dirak}
%ref: М, 352
{\footnotesize \paragraph{\footnotesize Справка.} \textsc{Число возможных распределений.}
  Задача о возможных распределениях $ N $ \textsl{неразличимых} дробинок по $ Z $ ячейкам
  \textsl{с запретом на повторение} (фермионы) 
  сводится к задаче о вытягивании $ Z $ шаров из урны с $ N $ шарами без
  возвращения, если
  мыслить себе задачу как выбор ячейки под дробинку, а не наоборот. Отсюда для
  Ферми -- Дирака получаем формулу $ W = C_Z^N $, а для Бозе -- Эйнштейна,
  аналогично, но с возвращением, $ W = C_{Z+N-1}^N $.

  Про фермионы см. \ref{princip-nerazlichimosti}.
}

\paragraph{Принцип Паули.}
\textit{Системы электронов (фермионов) встречаются в природе только в состояниях, описываемых антисимметричными волновыми функциями. }

Отсюда следует, что два одинаковых электрона (фермиона), входящих в одну систему, не могут находиться в одинаковых состояниях (иначе при перестановке волновая функция была бы чётной)\footnote{Однако отметим, что в одинаковом состоянии может находиться любое число бозонов.}.

\paragraph{Другая формулировка.} В одном и том же атоме не может быть более одного электрона с одинаковым набором четырёх квантовых чисел $n, l , m, m_s$ (про квантовые числа также см. \ref{princip-nerazlichimosti}).

\paragraph{Вывод распределения Ферми -- Дирака.}
Рассмотрим идеальный ферми-газ, то есть систему, состоящую из $N$ невзаимодействующих фермионов.
\begin{figure}[H]
	\centering
	\includegraphics[width=0.7\linewidth]{img/write-05/yacheiki}
	\caption{Возможное распределение ферми-частиц по ячейкам (черная точка - ферми частица, белая точка - отсутсвие частицы)}
	\label{fig:yacheiki}
\end{figure}
Разбив систему на $Z$ ячеек, найдем количество распределений $N$ фермионов по $Z$ ячейкам, то есть статистический вес макросостояния системы фермионов
\begin{equation*}
	W = C_Z^N = \frac{Z!}{N!(Z-N)!}.
\end{equation*}

Рассмотрим шестимерное фазовое пространство с координатами $x,y,z,p_x,p_y,p_z$. Разобьем его с помощью изоэнергетических поверхностей
\begin{equation*}
	f(x,y,z,p_x,p_y,p_z) \equiv E_i.
\end{equation*}
на тонкие слои так, что $|E_{i+1}-E_{i}|\ll E_i$. Пусть в пределы $i$-го слоя попадает $Z_i$ ячеек объемом $(2\pi\hbar)^3$ и $N_i$ частиц. Тогда аналогично примеру с раскладыванием фермионов по ячейкам получим статистический вес каждого энергетического слоя:
\begin{equation*}
	\Omega_i = \frac{Z_i!}{N_i!(Z_i-N_i)!}.
\end{equation*}
Статистический вес всей системы равен произведению статистических весов её подсистем:
\begin{equation*}
	\Omega = \prod_i \Omega_i.
\end{equation*}

Для нахождения наиболее вероятного распределения частиц по ячейкам, необходимо найти максимум статистического веса системы при условии, что количество частиц постоянно и полная энергия постоянна (жесткая адиабатическая оболочка), то есть
\begin{equation*}
	\sum N_i=\mathrm{const},\quad \sum E_iN_i=\mathrm{const}.
\end{equation*}

Вспомним формулу Больцмана для энтропии $S=k\ln \Omega$, вместо максимума $\Omega$ будем искать максимум энтропии. По свойствам логарифмов:
\begin{equation*}
	S=k\sum_i \big[\ln Z_i! - \ln N_i ! - \ln(Z_i - N_i)!\big].
\end{equation*}

Так как $N_i, Z_i \gg 1$, то воспользуемся формулой Стирлинга ($\ln n! \approx n\ln n - n$).
\begin{equation*}
	S=k\sum_i \big[
	Z_i \ln Z_i - Z_i - N_i \ln N_i + N_i - (Z_i-N_i)\ln(Z_i-N_i) + (Z_i-N_i)
	\big].
\end{equation*}
Слагаемое $Z_i \ln Z_i$ исключим из выражения, так как при поиске экстремума энтропии варьироваться будут только числа частиц в слое $N_i$, а данное слагаемое от них не зависит.

\textit{Метод множителей Лагранжа}. Рассмотрим функцию
\begin{equation*}
	F = S + \lambda_1 N + \lambda_2 E = -k\sum_i \big[N_i\ln N_i+(Z_i-N_i)\ln
  (Z_i-N_i)\big] + \lambda_1 \sum_i N_i + \lambda_2 \sum_i N_iE_i.
\end{equation*}
Приравнивая к нулю частные производные $F$ по $N_i$, получаем:
\begin{equation*}
	\frac{\partial F}{\partial N_i} = -k \left[
	\ln N_i + N_i\frac{1}{N_i}-\ln(Z_i-N_i)-(Z_i-N_i)\frac{1}{Z_i-N_i}
	\right] + \lambda_1 + \lambda_2 E_i = k \ln \frac{Z_i-N_i}{N_i} + \lambda_1 +
  \lambda_2E_i=0.
\end{equation*}
Отсюда следует, что:
\begin{equation*}
	\ln \frac{Z_i-N_i}{N_i} = -\frac{\lambda_2 E_i+\lambda_1}{k} \Leftrightarrow
	\frac{Z_i-N_i}{N_i} = \frac{1-N_i/Z_i}{N_i/Z_i}=\exp \left\{-\frac{\lambda_2
  E_i+\lambda_1}{k}\right\}.
\end{equation*}
Отношение $\frac{N_i}{Z_i}$ представляет собой среднее число фермионов $\langle
n_i \rangle$ в одной ячейке (т.е. в одном квантовом состоянии). Наиболее
вероятным значением $\langle n_i\rangle$, как следует из поиска экстремума, является:
\begin{equation*}
  \langle n_i\rangle =\frac{1}{\exp \left\{-\frac{\lambda_2
  E_i+\lambda_1}{k}\right\}+1}.
\end{equation*}

Поскольку все частные производные $F$ по $N_i$ равны нулю, то равен нулю и дифференциал этой функции $dF$, то есть:
\begin{equation*}
	dF = dS + \lambda_1 dN + \lambda_2 dE = 0.
\end{equation*}
Но так как число частиц системы постоянно ($\sum N_i=\mathrm{const}$), то
$dN=0$, а следовательно, варьируя\footnote{Какой-то физический приём,
конечно, $ E = \mathrm{const}$.}, получим
\begin{align*}
  dS + \lambda_2 dE = 0 &\implies \lambda_2 = -\frac{dS}{dE},\\
	\begin{cases}	
	dS \overset{\mathrm{def}}{=\joinrel=} \frac{\delta Q}{T},\\
  \delta Q = dE, & \text{т.к.}\, V=\mathrm{const}.
\end{cases}&\implies \lambda_2 = -\frac{dS}{dE} = -\frac{\delta Q}{TdE} =
  -\frac{1}{T}.
\end{align*}
Здесь первое уравнение системы есть следствие определения энтропии, а второе --- первый
закон термодинамики. $ \delta Q $ называют \emph{флунктуацией}.

Множитель $\lambda_1$ представим в виде $\lambda_1=\mu/T$, где $\mu$ ---
некоторая функция параметров состояния системы, в частности температуры. Можно
сказать, что таким образом мы определили $ \mu $, которую
называют \emph{химическим потенциалом}.

С учетом выражений для $\lambda_1,\,\lambda_2$, освобождаясь от индекса $i$, получаем
\begin{equation*}
  \langle n\rangle_{\textsc{Ф-Д}}=\frac{1}{\exp
  \left\{\frac{E-\mu}{kT}\right\}+1}.
\end{equation*}
распределение Ферми-Дирака. Оно определяет среднее количество фермионов, находящихся в квантовом состоянии с энергией $E$ при температуре $T$.

\begin{wrapfigure}{r}{0.5\textwidth}
	\centering
	\includegraphics[width=.8\linewidth]{img/write-05/fermi-dirak-T-notzero}
	\caption{Распределение Ферми-Дирака при $T\neq 0$}
	\label{fig:fermi-dirak-t-notzero}
\end{wrapfigure}

Химический потенциал $\mu$, очевидно, имеет размерность энергии, а в случае фермионов называется \textit{энергией Ферми} или \textit{уровнем Ферми}, обозначается $E_F$.

Анализируя выражение $\langle n\rangle_{\textsc{Ф-Д}}$ получим для $ T=0 $
\begin{align}
  \langle n\rangle_{\textsc{Ф-Д}} = \begin{cases}
    1, &E<E_F(0),\\
    0, &E>E_F(0),
	\end{cases}
\end{align}
то есть при $T=0$ распределение принимает вид ступеньки единичной высоты,
обрывающейся при $E=E_F(0)$. Для $T\neq 0$ см. рис \ref{fig:fermi-dirak-t-notzero}.
 \newpage
  \subsection{Сформулировать, какие частицы называются бозонами, привести примеры Бозе-частиц.
Сформулировать правило заполнения состояний Бозе- частицами. Используя формулу Больцмана
для энтропии, вывести распределение Бозе-Эйнштейна}


 \newpage
  \subsection{Описать явление термоэлектронной эмиссии. Рассматривая металл как потенциальный
ящик конечной глубины, заполненный «свободными» электронами, вывести формулу для плотности
тока насыщения $J_s$ термоэлектронной эмиссии.}
\label{sec:w7}
{\footnotesize \paragraph{\footnotesize Справка.} По определению \emph{дрейфовая
скорость} $ \vec v_d $ есть средняя векторная скорость всех частиц,
приобретаемая при воздействии электрического поля.

Концентрация частиц $ n $ есть их количество, делённое на объём (плотность для
тупых).
}

Термоэлектронная эмиссия -- излучение/испускание электронов из твердого тела при его нагреве.

Объектом исследования будет некий катод в вакууме, который кроме всего прочего мы нагреваем,
а вокруг него на некотором расстроянии находится анод -- в простом случае просто проводящая
пластина, с которой мы снимаем ток. Если рассматривать металл как потенциальный ящик конечной
глубины, получаем, что разность высоты потенциального барьера $U_0$ и энергии
Ферми $ E_F $ называется <<работой выхода>>
металла.
Током насыщения просто называется ток на аноде\footnote{Это практическая
  реализация идеи. Теоретически, нужно просто найти количество электронов,
испускаемых металлов и пролетающих некоторое расстояние при нагревании.}.

%TODO: идея про цилиндр \dot q? сколько заряда пролетает за dt на малой площадке
%(точке?)
Плотность тока насыщения\footnote{Здесь $ qn $ --- аналог плотности заряда, а
плотность, умноженная на скорость и есть плотность потока (можно нарисовать
цилиндр).} равна $\vec{J} = q n \vec{v_d} $, где $\vec{v_d} $ -- дрейфовая скорость, 
$n$ -- концентрация зарядов, $q$ -- заряд частицы. Ясно, что электроны некоторым образом
распределены по скоростям, а в этой формуле нас интересуют только энергии
$E > U_0 = E_F + A_B$. Соотвественно этому, $d J_x(v_x) = e v_x dn_e (v_x)$, где
$ dn_e(v_x) $ --- концентрация частиц, соответствующая скоростям в интервале $
(v_x, v_x + dv_x) $; а
\[
  dn_{e, [E, E+\Delta E]} \sim
  \dfrac{\sqrt{E}}{\exp\left(\frac{E-E_F}{kT}\right) +
  1} dE
\]
(см.
распределение свободных электронов в разделе \ref{sec:free-elecs}). Для
свободных электронов $E = p^2/(2m)$,
следовательно,
\[
  \sqrt{E} \sim p, \ dE \sim p dp \Rightarrow \sqrt{E} dE \sim p^2 dp \sim 4\pi p^2 dp = d \Omega_p,
\]
%TODO: разобраться с формулой объёма
где $d\Omega_p$ -- объём в фазовом пространстве $\vec{p} (p_x, p_y, p_z)$.
Тогда\footnote{Плотность при переходе к импульсу меняет только коэффициент.}
\[
  dn_{e, [p, p+\Delta p]} \sim \dfrac{d\Omega_p}{\exp\left(\frac{E(p) -
  E_F}{kT}\right) + 1}
\Rightarrow
dn_e (\vec{p}) = \dfrac{dp_x dp_y dp_z}{\exp\left(\frac{E(p) - E_F}{kT}\right) +
1}.
\]
Так как $v \sim p$, $\vec{v}$ сонаправлен с $\vec{p}$, то и $dn_e (v_x) \sim dn_e(p_x)$.
Учитывая также, что $E > U_0 = E_F + A_B, A_B \gg kT$, 
%TODO: почему??????
можно получить, что 
\[
  \exp\left(\dfrac{E-E_F}{kT}\right) \gg 1
  \Rightarrow
  dn_e (\vec{p}) \sim \exp\left(\dfrac{E_F}{kT}\right) \cdot \exp\left(\dfrac{-
  E(\vec{p})}{kT}\right) dp_x dp_y dp_z
\]
Дальше что-то очень простое:
\[
  E(\vec{p}) = E(p) = \dfrac{p_x^2 + p_y^2 + p_z^2}{2m_e}.
\]
Из этого найдём распределение $dn_e (p_x)$:
%TODO: расписать интегралы
\[
  dn_e (p_x) = \int_{\mathbb R^2} dn_e (\vec{p}) = e^{\dfrac{E_F}{kT}} \cdot 
  e^{- \dfrac{p_x^2}{2m_e kT}} dp_x \cdot 
  \int_\mathbb{R} e^{- \dfrac{p_y^2}{2m_e kT}} dp_y \cdot 
  \int_\mathbb{R} e^{- \dfrac{p_z^2}{2m_e kT}} dp_z
  \sim T e^{\dfrac{E_F}{kT}} \cdot e^{-\dfrac{p_x^2}{2m_e kT}} dp_x
\]
Объединяя все полученные результаты:
\begin{multline*}
  dJ_x \sim v_x \cdot dn_e (v_x) \sim p_x dn_e (p_x)
  \Rightarrow
  dJ_x \sim p_x T e^{\dfrac{E_F}{kT}} \cdot e^{-\dfrac{p_x^2}{2m_e kT}} dp_x
  \Rightarrow \\
  \Rightarrow
  J_x \sim T e^{\dfrac{E_F}{kT}} \int_{p_x, \min}^{+\infty} e^{-\dfrac{p_x^2}{2m_e kT}} p_x dp_x
  \sim T^2 e^{\dfrac{E_F}{kT}} \cdot e^{-\dfrac{(E_F + A_B)}{kT}}
  = T^2 e^{- \dfrac{A_B}{kT}}.
\end{multline*}
Здесь не очевидно было что такое $p_{x, \min}$, а это такой импульс, что энергия такой частицы
равна $U_0$, то есть $\dfrac{p_{x, \min}^2}{2m_e} = U_0$ (иные электроны просто
не вылетят).
 \newpage
  \subsection{Описать явление «холодной» эмиссии. Рассматривая металл как потенциальный ящик
конечной глубины заполненный «свободными» электронами, получить зависимость для плотности
тока $J$ «холодной» эмиссии от напряженности приложенного поля. Объяснить, почему холодная
эмиссия не может быть объяснена классической физикой}


 \newpage
  \subsection{Трансляционная симметрия твердых тел. Оператор трансляции и его собственные значения.
Сформулировать теорему Блоха (F. Bloch). Модель Кронига-Пени (R. Kronig, W. Penney) и ее
решение. Показать, что полученный спектр энергий представляет собой совокупность зон.
Заполнение зон в металлах, полупроводниках и диэлектриках}


 \newpage
  \subsection{Электропроводность $\sigma$ чистых полупроводников. Подвижность $\mu$ носителей заряда.
Эффективная масса $m^*$. Вывести формулы для концентрации $n_e$ электронов и дырок $n_h$
в чистых полупроводниках. Найти положение уровня Ферми в чистых полупроводниках при $T=0$.
Температурная зависимость собственной проводимости полупроводников.}

\paragraph{Эффективная масса} рассмотрим группу электронов полуклассически: $p = \hbar k$, тогда
групповая скорость:
\[
  v_\text{гр} = \dfrac{\partial \omega}{\partial k} = \dfrac{\partial E}{\partial p} 
  = \dfrac{\partial E}{\partial k} \left( \dfrac{\partial p}{\partial k}  \right)^{-1}
  = \dfrac{1}{\hbar} \dfrac{\partial E}{\partial k} 
\]
в полуклассическом рассмотрении: <<$\vec{F} = \dfrac{\partial \vec{p}}{\partial t},
\vec{a} = \dfrac{\partial \vec{v}}{\partial t}$>>. Тогда:
\[
  \dfrac{\partial v_\text{гр}}{\partial t}
  = \dfrac{1}{\hbar} \dfrac{\partial }{\partial t} \left( \dfrac{\partial E}{\partial k} \right)
  = \dfrac{1}{\hbar} \dfrac{\partial }{\partial k} \left( \dfrac{\partial E}{\partial k}  \right) \cdot \dfrac{\partial k}{\partial t} 
  = \dfrac{1}{\hbar^2} \dfrac{\partial^2 E}{\partial k^2} \cdot \dfrac{\partial p}{\partial t}
  = a 
\]
сопоставляя определение силы и это выражение, получаем, что эффективная масса $m^* = \hbar^2 / \dfrac{\partial^2 E}{\partial k^2}$. Такая масса учитывает, что электрон/дырка
находятся в периодическом кристалле.

\paragraph{Электропроводность твердых тел} $\vec{J} = en \vec{v}_d$, $\vec{v}_d = \vec{a} \cdot \tau$, где $\tau$ -- время <<релаксации>> между соударениями с рассеивающими элементами. Тогда:
\begin{equation}\label{write:10:v_d}
  \vec{a} = \dfrac{\vec{F}}{m^*} = \dfrac{e\vec{E}}{m^*}
  \Rightarrow
  \vec{J} = \dfrac{e^2 \tau}{m^*} n \vec{E}
  \Rightarrow
  \vec{v}_d = \dfrac{e \tau}{m^*} \vec{E} = \mu \vec{E}
\end{equation}
$\mu$ называют подвижностью. $\vec{J} = \dfrac{1}{\rho_0} \vec{E} = \sigma \vec{E}$ -- $\rho$ --
удельное сопротивление, $\sigma$ -- удельная проводимость.

\paragraph{Концентрация электронов и дырок} концентрация свободных электронов:
\[
  dn_e = f_\text{ФД} (E) dN_e = f_\text{ФД} \cdot g(E) dE, g(E) \sim \sqrt{E}.
\]
Сдвинем энергии так, чтобы 0 энергии приходился на верхнюю часть зоны валентности. 
Нас интересуют только электроны, находящиеся в зоне проводимости, то есть энергия которых
больше ширины запрещённой зоны $E_g$. Концентрацию электронов можно найти путём интегрирования:
$n_e = \int dn_e$, однако, используем также то, что $E-E_F \gg kT$, тогда в знаменателе у
распределения Ф-Д можно пренебречь единицей. Тогда:
\[
  n_e \sim \int_{E_g}^{+\infty} \sqrt{E-E_g} e^{-\dfrac{E}{kT}} dE \sim \dots
  \sim (m_e^*)^{3/2} T^{3/2} \cdot e^{\dfrac{E_F - E_g}{kT}}
\]
У дырок наоборот, рассматривается только энергии меньше нуля, для них так же верно
$E<0 \Rightarrow |E-E_F| \gg kT$, тогда
$1 - f_\text{ФД} = 1 - \dfrac{1}{e^{\dots} + 1} \approx 1 - \left( 1 - e^{\dots} \right) = e^{\dfrac{E-E_F}{kT}} $. 
\[
  n_h \sim \int_{-\infty}^0 \sqrt{-E} \cdot e^{\dfrac{E}{kT}} dE \sim (m_h^*)^{3/2} T^{3/2} e^{-\dfrac{E_f}{kT}}
\]

\paragraph{Положение уровня Ферми в чистых полупроводниках}

Когда полупроводник чистый, концентрация электронов в зоне проводимости равна концентрации электронов в зоне валентности:
\[
  n_e = n_h \Leftrightarrow (m_e^*)^{3/2} T^{3/2} e^{\dfrac{E_G - E_g}{kT}} = (m_h^*)^{3/2} T^{3/2} e^{-\dfrac{E_F}{kT}} 
  \Leftrightarrow
  E_F(T) = \dfrac{3}{4} kT \ln (\dfrac{m_h^*}{m_e^*}) + \dfrac{E_g}{2}
\]
при $T = 0$, соотвественно, уровень Ферми находится ровно по середине запрещённой зоны.

\paragraph{Температурная зависимость}
\[
  \sigma = e^2 \left( \dfrac{n_e \tau_e}{m_e^*} + \dfrac{n_h \tau_h}{m_h^*} \right) 
\]
при изменении температуры качественно меняется только время релаксации. Оценим его из классических соображений. Рассмотрим цилиндр небольшого радиуса, который окружает траекторию рассматриваемого электрона. Электрон движется с некоторой усредненной скоростью $<v>$, поэтому рассмотрим цилиндр длиной $<v> \delta t$. Количество столкновений $N$ с рассеивающими частицами пропорционально количеству частиц в этом цилиндре, а оно в свою очередь пропорционально концентрации рассеивающих частиц. Среднее время между столкновениями: $\tau = \dfrac{t}{N} \sim \dfrac{1}{n <v>}$. Рассеиваются наши электроны на фононах, которые являются бозонами, поэтому концентрация подчиняется статистике Б-Э. Поэтому:
\[
  \tau \sim \dfrac{1}{n_\text{Ф} <v>} \sim \dfrac{e^{\dfrac{E_0}{kT}} - 1}{v}
  \sim \dfrac{1}{T^{1/2} T} \sim T^{-3/2}
\]
а проводимость в таком случае:
\[
  \sigma \sim n_e \tau \sim T^{3/2} e^{-E_g / (kT)} T^{-3/2} \sim e^{-E_g / (kT)}
\]
 \newpage

  \newpage

  \section{Устная часть}
  \subsection{Фундаментальные понятия и принципы Квантовой Механики}

\subsubsection{Волны Де Бройля}

\begin{multicols}{2}
	
\paragraph{Общий случай} Плоская волна частотой $\omega$, распространяющаяся вдоль оси $x$:
\begin{equation*}
	\xi(x,t) = A \exp \left\{-i(\omega t -kx)\right\},
\end{equation*}
где $A$ - амплитуда, $k=\frac{2\pi}{\lambda}$ - волновое число.

\paragraph{Гипотеза Де Бройля} 
\thispagestyle{empty}
Свободной частице с энергией $E$ и импульсом $p$, движущейся вдоль $x$, соответсвует плоская волна:
\begin{equation*}
	\Psi(x, t) = A \exp \left\{-\frac{i}{\hbar}(E t - px)\right\}
\end{equation*}
Посмотрев на выражения волновых процессов в общем виде $\xi$ и для волны Де Бройля $\Psi$, заметим некоторые соотношения:
\begin{itemize}
	\item частота $\omega = \frac{E}{\hbar}$
	\item длина волны Де Бройля $\lambda_{\text{Б}}=\frac{2\pi\hbar}{p}$
	\item энергия $E = \hbar \omega = h\nu$
	\item $\vec p = \hbar \vec k$, где $\vec k$ - волновой вектор
\end{itemize}

\paragraph{Свойства волн Де Бройля}
\begin{enumerate}[label=\textbf{№~\arabic{enumi}}]
	\item В процессе распространения волны могут отражаться, преломляться, интерферировать и дифрагировать по обычным волновым законам
	
	\item Фазовая скорость $v_\text{фаз}$ - скорость, с которой распространяются точки волны с постоянной фазой.
	Выражение для фазовой скорости вытекает из условия постоянности фазы при его дифференцировании:
	\begin{equation*}
		Et-px=\mathrm{const} \overset{\dfrac{d}{dt}}{\longrightarrow} v_\text{фаз} = \frac{dx}{dt} = \frac{E}{p}.
	\end{equation*}
	При подстановке известных соотношений, а именно $E=mc^2$ и $p=mv$ в $v_\text{фаз}$, получим
	\begin{equation*}
		v_\text{фаз} = \frac{c^2}{v}
	\end{equation*} выражение для фазовой скорости\footnote{$v<c \implies v_\text{фаз} = \frac{c^2}{v} > c$ --- это не противоречит СТО. Ограничения, накладываемые СТО касаются скорости переноса массы/энергии, но фазовая скорость волны ничего из этого не характеризует.}.
	
	\item Групповая скорость $v_\text{гр}$
	\begin{equation*}
		v_\text{гр} = \frac{dw}{dk} = \frac{d(\hbar\omega)}{d(\hbar k)} = \frac{dE}{dp}
	\end{equation*}
	Возьмем известное всем выражение из теории относительности:
	\begin{equation*}
		E^2=p^2c^2+E_0^2=\left[E_0=m_0c^2\right]=E^2=p^2c^2+m_0^2c^4
	\end{equation*}
	Продифференцируем по t:
	\begin{equation*}
		2EdE=2pc^2dp\longrightarrow\frac{dE}{dp}=\frac{pc^2}{E}
	\end{equation*}
	Воспользовавшись этим знанием, получаем, что групповая скорость волны равна скорости движения частицы:
	\begin{equation*}
		v_\text{гр} = \frac{pc^2}{E} = \frac{pc^2}{mc^2} = \frac{p}{m} = v
	\end{equation*}
	
	\item Длина волны Де Бройля для нерелятивистских $(v\ll c)$ и релятивистких ($v\approx c$) частиц.
	\begin{itemize}
		\item Нерелятивисткий случай $(v\ll c)$:
		\begin{equation*}
		E_k=\frac{mv^2}{2}=\frac{p^2}{2m_0}\implies p=\sqrt{2m_0E_k}
		\end{equation*}
		\begin{equation*}
		\lambda_\text{Б}=\frac{2\pi\hbar}{p}=\frac{2\pi\hbar}{\sqrt{2m_0E_k}}
		\end{equation*}
		\item Релятивистский случай $(v\approx c)$:
		\begin{equation*}
		p=\frac{1}{c}\sqrt{E_k(E_k+2m_0c^2)} \Leftrightarrow
		\end{equation*}
		\begin{equation*}
		\Leftrightarrow p=\sqrt{2m_0E_k}\sqrt{1+\frac{E_k}{2m_0c^2}}
		\end{equation*}
		\begin{equation*}
			\lambda_\text{Б}'=\frac{2\pi\hbar}{p}=\frac{2\pi\hbar}{\sqrt{2m_0E_k}\sqrt{1+\frac{E_k}{2m_0c^2}}}
		\end{equation*}
		В этом выражении можно выделить нерелятивистскую дебройлевскую длину волны $\lambda_\text{Б}$, тогда 
		\begin{equation*}
			\lambda_\text{Б}'=\frac{\lambda_\text{Б}}{\sqrt{1+\frac{E_k}{2m_0c^2}}}
		\end{equation*}
	\end{itemize}
\end{enumerate}

\end{multicols}
\subsubsection{Соотношения неопределенностей Гейзенберга}
\begin{multicols}{2}
	\paragraph{Смысл}
	Двойственная корпускулярно-волновая природа микрочастиц накладывает ограничения на точность определения значений физических величин, характеризующих состояние частицы. Причем эти ограничения никак не связаны с точностью измерений, достижимой в конкретном эксперименте, а имеют принципиальное значение.
	\paragraph{Выражения для неопределенностей}
	\begin{equation*}
		\begin{cases}
		\left.\begin{array}{lll}
			\Delta p_x \Delta x & \geq \dfrac{\hbar}{2} &  \\[10pt]
			\Delta p_y \Delta y & \geq \dfrac{\hbar}{2} &  \\[10pt]
			\Delta p_z \Delta z & \geq \dfrac{\hbar}{2} &  
		\end{array}\right] \text{иногда пишут $\hbar$} \\[1.5cm]
		\begin{array}{rr}
			\Delta E \Delta t  & \geq \hbar 
		\end{array}
		\end{cases},
	\end{equation*}где $\Delta P_{x}$ --- неопределенность импульса $P_{x}$, \\
	$\Delta x$ --- неопределенность координаты $x$,\\
	$\Delta E$ --- неопределенность энергии,\\
	$\Delta t$ (иногда пишут $\tau$) --- среднее \textit{время жизни} в данном энергетическом состоянии.
	
	\paragraph{Общий ход вывода соотношений для моментов и координат}
	\begin{figure}[H]
		\centering
		\includegraphics[width=.9\linewidth]{img/oral-01/electron-difraction}
		\caption{Картина дифракции электрона на щели}
		\label{fig:electron-difraction}
	\end{figure}
	(см Рис. \ref{fig:electron-difraction}) Пусть падающий электрон обладает импульсом $\vec p_0$. По гипотезе Де Бройля поставим в соответствие этому электрону плоскую волну с волновым вектором $\vec k = \frac{\vec p_0}{\hbar}$ и длиной волны $\lambda_{\text{Б}}=\frac{2\pi\hbar}{p_0}$.
	
	До прохождения щели известен импульс электрона $p_x=p_z=0,\, p_y=p_0$, а его координата $x$ неизвестна.
	
	При прохождении щели неопределенность координаты $x$ становится равной $\Delta x$, появляется неопределенность импульса $\Delta p_x$, обусловленная дифракцией электронов на щели\footnote{Электроны после прохождения щели описываются теперь не плоской, а расходящейся волной, интенсивность которой зависит от угла дифракции $\varphi$.}. Меняется также и проекция $p_x$ импульса электрона на ось $x$.
	
	Центральный дифракционный максимум, в который попадет большинство электронов, описывается углом $\varphi_1$, задающим первый минимум интенсивности. Из теории дифракции ($\Delta x \sin(\varphi_k)=k\lambda,\,k\in\mathbb{N}$) запишем уравнение для нахождения $\varphi_1,\,(k=1)$:
	\begin{equation*}
		\Delta x \sin \varphi_1=\lambda_{\text{Б}}
	\end{equation*}
	За счет малости $\varphi_1$ используем приближение:
	\begin{equation*}
		\frac{\lambda_{\text{Б}}}{\Delta x} = \sin \varphi_1 \approx \tg \varphi_1
	\end{equation*}
	В то же время угол $\varphi_1$ можно выразить через проекции $p_x,\,p_y$ импульса электрона:
	\begin{equation*}
		\tg \varphi_1 = \frac{p_x}{p_y} = \frac{p_x}{p_0}
	\end{equation*}
	Считая, что неопределенность $\Delta p_x$ сравнима с $p_x$, получаем:
	\begin{equation*}
		\tg \varphi_1 \approx \frac{\Delta p_x}{p_0}
	\end{equation*}
	Отсюда следует, что:
	\begin{equation*}
		\tg \varphi_1 \approx \frac{\Delta p_x}{p_0} \approx \frac{\lambda_{\text{Б}}}{\Delta x}
	\end{equation*}
	Окончательно получаем:
	\begin{equation*}
		\Delta x \Delta p_x \approx \lambda_{\text{Б}}p_0
	\end{equation*}
	Поскольку $\lambda_{\text{Б}}=\frac{2\pi\hbar}{p_0}$, то с учетом сделанных приближений и упрощений:
	\begin{equation*}
		\Delta x \Delta p_x \approx \underset{\text{пишут по разному}}{2\pi\hbar \geq \hbar \geq \frac{\hbar}{2}}
	\end{equation*}
	
	Поскольку ось $x$ физически ничем не была выделена, то аналогичное соотношение оказывается справедливым для других координатных осей $y$ и $z$.
\end{multicols}

\subsubsection{Постулаты квантовой механики}

\subsubsection{Принцип суперпозиции квантовых состояний}

\subsubsection{Принцип неразличимости тождественных частиц}
Состояние электрона в атоме определяется набором четырёх квантовых чисел:
\begin{table}[H]
	\begin{center}
		\begin{tabular}{c|r|l}
			1 &            главное & $n\in\mathbb{N}$                                \\[5pt]
			2 &        орбитальное & $l\in[0,n-1]$                                   \\[5pt]
			3 &          магнитное & $m\in[-l,l]$                                    \\[5pt]
			4 & магнитное спиновое & $m_s\in\left\{\frac{-1}{2},\frac{1}{2}\right\}$ \\[5pt]
		\end{tabular}
	\end{center}
	\caption{Квантовые числа}
\end{table}
\paragraph{Принцип неразличимости тождественных частиц} Тождественные частицы экспериментально различить невозможно.
\paragraph{Математическая запись}
\begin{equation*}
	\left|\Psi(x_1,x_2)\right|^2 = \left|\Psi(x_2,x_1)\right|^2,
\end{equation*}
где $x_1$ и $x_2$ --- соответственно совокупность пространственных и спиновых
координат первой и второй частиц. Возможны два случая:
\begin{table}[H]
	\begin{center}
	\begin{tabular}{|l|c|c|}
		\hline
		Случай   &         симметричный          &        антисимметричный        \\ \hline
		Запись   & $\Psi(x_1,x_2)=\Psi(x_2,x_1)$ & $-\Psi(x_1,x_2)=\Psi(x_2,x_1)$ \\ \hline
		Спин     &      $m_s\in\mathbb{Z}$       & $m_s\in\mathbb{Z}+\frac{1}{2}$ \\ \hline
		Название &         \text{бозоны}         &        \text{фермионы}         \\ \hline
		Примеры  &     $\pi$-мезоны, фотоны      &  электроны, протоны, нейтроны  \\ \hline
	\end{tabular}
	\end{center}
\end{table}
 \newpage
  \subsection{Следствия из основных положений}

\subsubsection{Уравнение непрерывности для плотности вероятности, как следствие основных
постулатов и уравнения Шрёдингера (вывод)}

Уравнение Шредингера:
\[
  -i \hbar \dfrac{\partial \Psi}{\partial t} = - \dfrac{\hbar^2}{2m} \nabla^2 \Psi + U \Psi,
\]
комплексно сопряженное уравнение:
\[
  -i \hbar \dfrac{\partial \Psi^*}{\partial t} = - \dfrac{\hbar^2}{2m} \nabla^2 \Psi^* + U \Psi^*,
\]

Умножим обычное на $\Psi^*$, а компескно сопряженное умножим на $\Psi$, потом сложим:
\[
  i \hbar \left( \Psi^* \dfrac{\partial \Psi}{\partial t} + \Psi \dfrac{\partial \Psi^*}{\partial t} \right)
  = - \dfrac{\hbar^2}{2m} \left( \Psi^* \nabla^2 \Psi - \Psi \nabla^2 \Psi^* \right) 
  \Rightarrow
  \left( \Psi^* \dfrac{\partial \Psi}{\partial t} + \Psi \dfrac{\partial \Psi^*}{\partial t} \right)
  = \dfrac{i \hbar}{2m} \left( \Psi^* \nabla^2 \Psi - \Psi \nabla^2 \Psi^* \right) 
\]
левая часть равенства представляет собой $\dfrac{\partial \Psi^* \Psi}{\partial t} =
\dfrac{\partial |\Psi|^2}{\partial t}$, то есть производную плотности вероятности $\rho_p$
по времени. В правой части преобразуем:
\[
  \left( \Psi^* \nabla^2 \Psi - \Psi \nabla^2 \Psi^* \right)
  = \nabla \left( \Psi^* \nabla \Psi - \Psi \nabla \Psi^* \right)
\]
(это выражение проверяется в лоб). Обозначая $ - \dfrac{i \hbar}{2m} \left( \Psi^* \nabla \Psi - \Psi \nabla \Psi^* \right) = \vec{J}_p$, окончательно получаем:
\[
  \dfrac{\partial \rho_p}{\partial t} = - \nabla \vec{J}_p =  - \operatorname{div} \vec{J}_p.
\]
Это выражение называется уравнением непрерывности для плотности вероятности.

\subsubsection{Симметричные и антисимметричные состояния}

Для системы нескольких частиц, волновая функция представляется в виде: $\Phi( \vec{q}_1, \vec{q}_2, \dots, \vec{q}_n )$. Рассмотрим оператор перестановки $\hat{P}_{ij}$:
\[
  \hat{P}_{ij} \Psi(\vec{q}_1, \dots, \vec{q}_i, \dots, \vec{q}_j, \dots, \vec{q}_n)
  = \Psi(\vec{q}_1, \dots, \vec{q}_j, \dots, \vec{q}_i, \dots, \vec{q}_n)
\]
рассмотрим спектр этого оператора:
\[
  \hat{P}_{ij} \Psi = \lambda \Psi, \hat{P}^2_{ij} \Psi = \Psi = \lambda^2 \Psi 
  \Rightarrow
  \lambda = \pm 1.
\]
Функции $\Psi$, у которых $\lambda = 1$, называются симметричными, а другие -- антисимметричными. Причём несложно доказать, что если системе одинаковых частиц в некоторый момент времени соответствовала функция (анти) симметричная, то и в последующие моменты времени она останется такой же:
\[
  - i \hbar \dfrac{\partial \Psi}{\partial t} = \bar{H} \Psi
\]
ну и вот справа будет функция такой же симметричности как и $\Psi$, поэтому и слева тоже.

\subsubsection{Принцип Паули в строгой формулировке}

В одном и том же квантовом состоянии может находится не более одного фермиона. Состояние можно
характеризовать 4 величинами $L_1, L_2, L_3, L_s$: все эти величины одновременно измеримы. Первые
3 величины относятся к движению центра масс и незваисимы друг от друга, $L_s$ описывает состояние
спина электрона. 

Пример такой совокупности величин: $E_{n, l} = L_1; L^2 = L_2, L_z = L_3; L_s = m_s$.

Тогда принцип Паули можно также переформулировать следующим образом: в процессе измерения над 
системой каждая четвёрка величин $L_1, L_2, L_3, L_4$ может быть получена не более одного раза.

\subsubsection{Нахождение среднего значения физической величины (вывод)}

Пусть некоторой физической величине $f$ соотвествует оператор $\hat{f}$. Найдём его собственные значения и обозначим их $f_n$, а собственные функции -- $\Psi_n$.

Тогда любое состояние $\Psi$ раскладывается по базису из собственных функций этого оператора:
\[
  \Psi = \sum_n c_n \Psi_n
\]
это утверждение носит название 3-его постулата КМ. Причём вероятность $P(f = f_n)$ равняется квадрату модуля коэффициента перед соотвествующей собственной функцией:
\[
  P(f = f_n) = P_n = c_n^* c_n.
\]

Тогда среднее значение (матожидание) некоторой величины $f$ выражается:
\begin{multline*}
  <f> = \sum_n P_n f_n
  = \sum_n c_n^* c_n f_n
  = \sum_n c_n^* f_n \int \Psi_n^* \Psi dV
  = \sum_n c_n^* \int \Psi \hat{f} \Psi_n^*
  = \int \Psi (\hat{f} \sum_n c_n \Psi_n)^* dV = \\
  = \int \Psi (\hat{f} \Psi)^* dV
  = \int \Psi^* (\hat{f} \Psi) dV
\end{multline*}

\subsubsection{Дискретность спектра энергий для связанных состояний квантовых систем
(Объяснить связь с основными постулатами)}

Связным состоянием называется такое состояние, в котором частица движется в ограниченной области
пространства $\Omega$, следовательно, граничное условие для $\Psi$:
$\left. \Psi \right|_{\partial \Omega} = 0$. В таком случае спектр энергий квантуется.

Из ландафщица:

Спектр собственных значений энергии может быть как дискретным, так и непрерывным.
Стационарное состояние дискретного спектра всегда соответствует финитному движению
системы, т. е. движению, при котором система или какая-либо ее часть не уходит на
бесконечность. Действительно, для собственных функций дискретного спектра интеграл
$\int |\Phi|^2 dq$, взятый по всему пространству, конечен. Это, во всяком случае,
означает, что квадрат $|\Phi|^2$ достаточно быстро убывает, обращаясь на бесконечности в
нуль. Другими словами, вероятность бесконечных значений координат равна нулю, т. е.
система совершает финитное движение или, как говорят, находится в связанном состоянии.

Для волновых функций непрерывного спектра интеграл $\int | \Phi |^2 dq$ расходится.
Квадрат волновой функции $|\Phi|^2$ не определяет здесь непосредственно вероятности
различных значений координат и должен рассматриваться лишь как величина, пропорциональная
этой вероятности. Расходимость интеграла $\int | \Phi |^2 dq$ всегда бывает связана с тем,
что $|\Phi|^2$ не обращается на бесконечности в нуль (или обращается в нуль недостаточно
быстро). Поэтому можно утверждать, что интеграл $\int | \Phi |^2 dq$, взятый по
области пространства, внешней по отношению к любой сколь угодно большой, но конечной
замкнутой поверхности, будет все же расходиться. Это значит, что в рассматриваемом
состоянии система (или какая-либо ее часть) находится на бесконечности. Для волновой функции,
представляющей собой суперпозицию волновых функций различных стационарных состояний
непрерывного спектра, интеграл $\int |\Phi|^2 dq$ может оказаться сходящимся, так что
система находится в конечной области пространства. Однако с течением времени эта область
будет неограниченно смещаться, и в конце концов система уходит на бесконечность.


\subsubsection{Квантовое туннелирование и его проявление в различных явлениях ($\alpha$-распад, холодная эмиссия, тунельный микроскоп)}

Определение. Туннельный эффект или туннелирование - это явление преодоления микрочастицей
потенциального барьера в случае, когда её полная энергия (остающаяся при туннелировании
неизменной) меньше высоты барьера.

\paragraph{Альфа распад}
Альфа-частица испытывает туннельный переход через потенциальный барьер, обусловленный
ядерными силами, поэтому альфа-распад является существенно квантовым процессом.

\paragraph{Холодная эмиссия}


\paragraph{Тунельный микроскоп}
Рассматривать отдельные атомы можно с помощью устройства, использующего квантовый эффект
туннелирования – сканирующий туннельный микро- скоп (СТМ). Точнее, сканирующий туннельный
микроскоп не рассматривает, а как бы «ощупывает» ис- следуемую поверхность. Очень тонкая
игла-зонд с острием толщиной в один атом перемещается над поверхностью объекта на расстоянии
порядка одного нанометра. При этом согласно законам квантовой механики, электроны преодолевают
вакуумный барьер между объектом и иглой – туннелируют, и между зондом и образцом начинает
течь ток. Сила этого тока очень сильно зависит от расстояния между концом иглы и поверхностью
образца – при изменении зазора на десятые доли нанометра сила тока может возрасти или
уменьшиться на порядок. Так что, перемещая зонд вдоль поверхности с помощью пьезоэлементов
и отслеживая изменение силы тока, можно исследовать ее рельеф практически «на ощупь».

 \newpage
  \subsection{Элементы статистической физики}
\subsubsection{Принцип детального равновесия и его применение (вывод закона Кирхгофа, связь $r_{T, \omega}^*$ vs $u_{T, \omega}$, формула Эйнштейна для коэффициентов A и B)}
\paragraph{Формула Эйнштейна для коэффициентов A и B} см. письменный вопрос \ref{einstein-a-b}

\subsubsection{Плотность квантовых состояний и плотность мод колебаний электромагнитного поля в полости}
Найдём число квантовых состояний, по которым могут распределяться частицы, при
условии, что энергия этих состояний не превышает некоторого значения $ E $.
Пусть сначала частица находится в трёхмерной потенциальной яме с непроницаемыми
стенками. Энергия в такой яме равна 
\[
  E = \frac{\pi^2\hbar}{2m_0} \left[ \left( \frac{n_1}{a_1} \right)^2 + \left(
  \frac{n_2}{a_2}\right)^2 + \left( \frac{n_3}{a_3} \right)^2   \right],
\]
где $ a_1 $, $ a_2 $ и $ a_3 $ --- стороны прямоугольного параллелепипеда, а $
n_1 $, $ n_2 $, $ n_3 = 1, 2, 3,\ldots$ --- квантовые числа. Пусть $ E $ столь
велико, что спектр можно считать практически непрерывным.

В дискретном трёхмерном пространстве квантовых чисел введём обозначение  
\[
  r^2 = \frac{(n_1a_2a_3)^2 + (n_2a_1a_3)^2 + (n_3a_1a_2)^2}{(a_1a_2a_3)^{4/3}}
\]
и перепишем выражение для энергии в виде 
\[
  E = \frac{\pi^2\hbar^2}{2m_0(a_1a_2a_3)^{2/3}}r^2,
\]
откуда 
\[
  r = \frac{(a_1a_2a_3)^{1/3}\sqrt{2m_0E}}{\pi\hbar}.
\]

Рассмотрев теперь сферу (её восьмую часть) с радиусом $ r $, найдём число
возможных состояний как объём этой фигуры:
\[
  G = \frac{1}{6}\pi r^3 J_z = \frac{1}{6} \pi \frac{ \left( \sqrt{2m_0E}
  \right)^3 }{\pi^3\hbar^3}J_z a_1a_2a_3,
\]
где $ J_z $ есть количество возможных проекций спина частицы (для электрона,
например, они принимают значения $ \pm 1/2 $, откуда $ J_z = 2 $).
Действительно, каждой точке пространства соответствует $ J_z $ состояний.
Поскольку $ a_1a_2a_3 $ представляет собой объём потенциальной ямы $ V $, а $
\sqrt{2m_0E} $ есть нерелятивистский импульс частицы $ p $, то соотношение можно
переписать в виде 
\[
  G = \frac{4}{3} \pi \rho^3 V \frac{J_z}{(2\pi\hbar)^3}.
\]
При этом $ (4\pi p^3)/3 $ есть не что иное, как объём шара радиусом $ p $
--- импульса, соответствущего максимальной энергии $ E $ --- в пространстве
импульсов $ p_x $, $ p_y $, $ p_z $. Таким образом, выражение для $ G $ в
фазовом пространстве $ x $, $ y $, $ z $, $ p_x $, $ p_y $, $ p_z $ имеет вид 
\[
  G = \frac{V_{\text{фаз}}}{(2\pi \hbar)^3} J_z,
\]
то есть $ G $ пропорционально фазовому объёму.

Оказывается, что данный результат справедлив для ям произвольной формы.

Таким образом, объём фазового пространства, приходящийся на одно квантовое
состояние, равен $ (2\pi\hbar)^3 = h^3 $. Запишем это следующим образом: 
\[
  \Delta x \Delta y \Delta z \Delta p_x \Delta p_y \Delta p_z = h^3,
\]
где $ \Delta x, \ldots \Delta p_x, \ldots $ --- размеры ячейки в фазовом пространстве, приходящейся
на одно состояние. При этом благодаря равноправности координат, например, $
\Delta x \Delta
p_x = 2\pi \hbar$. Этот результат, как легко видеть, согласуется с принципом 
неопределенности. Действительно, размеры ячейки фазового 
пространства, приходящейся на одно состояние, должны определяться
теми ограничениями на значения координаты и импульса, которые
накладывают соотношения неопределенностей.

Найдём теперь \emph{плотность квантовых состояний} $ g(E) $, то есть число
состояний, приходящихся на единичный интервал энергий. Найдём его, переписав в
виде 
\[
  g(E)  = \frac{dG}{dp} \frac{dp}{dE} = J_z \frac{dp}{dE} \frac{d}{dp} \left(
  \frac{4}{3} \frac{\pi p^3 V}{(2\pi \hbar)^3}\right) = J_z \frac{4\pi p^2
  V}{(2\pi\hbar)^3} \frac{dp}{dE}.
\]
Данное выражение является общим, то есть справедливым для любых частиц.

\subsubsection{Характерные свойства теплового излучения в полости, как следствие его равновесности}

 \newpage
  \subsection{Элементы физики твердого тела}

\subsubsection{Трансляционная симметрия --  определение, примеры объектов с трансляционной симметрией}

\subsubsection{Функция распределения для свободных электронов (вывод). Связь энергии Ферми и концентрации электронов}

Будем рассматривать металл как потенциальный ящик с абсолютно непроницаемыми стенками, в 
котором находятся электроны, не взаимодействующие между собой. Рассмотрим число квантовых состояний 
$N_E$, энергия которых меньше $E$ ($l_i$ -- размер коробки по $i$-ой координате):
\[
  E_{n_1, n_2, n_3} = \dfrac{\pi^2 \hbar^2}{2 m_e} \left(
    \left( \dfrac{n_1}{l_1} \right)^2 +
    \left( \dfrac{n_2}{l_2} \right)^2 +
    \left( \dfrac{n_3}{l_3} \right)^2 \right) < E
\]
обозначая: $V = l_1 l_2 l_3 = l_0^3$, $\alpha_i = \dfrac{l_i}{l_0}$, получаем:
\[
  \dfrac{\pi^2 \hbar^2}{2 m_e l_0^3} \left(
    \left( \dfrac{n_1}{\alpha_1} \right)^2 +
    \left( \dfrac{n_2}{\alpha_2} \right)^2 +
    \left( \dfrac{n_3}{\alpha_3} \right)^2 \right) < E
\]
обозначим $E_0 = \dfrac{\pi^2 \hbar^2}{2 m_e l_0^3}$, $\varepsilon = \dfrac{E}{E_0}$:
\[
  \dfrac{1}{\varepsilon} \left(
    \left( \dfrac{n_1}{\alpha_1} \right)^2 +
    \left( \dfrac{n_2}{\alpha_2} \right)^2 +
    \left( \dfrac{n_3}{\alpha_3} \right)^2 < 1
  \Leftrigtharrow
  \left( \dfrac{n_1}{\alpha_1 \sqrt{\varepsilon}} \right)^2 +
    \left( \dfrac{n_2}{\alpha_2 \sqrt{\varepsilon}} \right)^2 +
    \left( \dfrac{n_3}{\alpha_3 \sqrt{\varepsilon}} \right)^2 < 1
\]
Таким образом, приходим к тому, что нужно найти объём в пространстве квантовых чисел
(квазинепрерывном), это уравнение эллипса, поэтому объем равен 1/8 части объёма эллипса
(учитывая то, что каждое кв число больше 0):
\[
  \Omega_E = \dfrac{1}{8} \dfrac{4}{3} \pi (\sqrt{\varepsilon})^3 \alpha_1 \alpha_2 \alpha_3
  = \dfrac{1}{6} \pi \varepsilon^{3/2}.
  \Rightarrow
  N_\text{Я} = \dfrac{1}{6} \pi \varepsilon^{3/2}
  = \dfrac{1}{6} \pi E^{3/2} \left( \dfrac{2 m_e l_0^3}{3 \pi^2 \hbar^2} \right)^{3/2}
  = \dfrac{\sqrt{2} m_e^{3/2}}{3 \pi^2 \hbar^3} E^{3/2} V
\]
с учётом спина:
\[
  N_E = 2 N_\text{Я} = \dfrac{2 \sqrt{2} m_e^{3/2}}{3 \pi^2 \hbar^3} E^{3/2} V 
\]
перейдём к концентрации:
\[
  \tilde n_E = \dfrac{N_E}{V} = \dfrac{2 \sqrt{2} m_e^{3/2}}{3 \pi^2 \hbar^3} E^{3/2}.
\]
Это количество состояний, для которых энергия меньше $E$, для получения концентрации состояний в
отрезке $[E, E+\Delta E]$, продифференцируем это:
\[
  dn_E = \dfrac{\sqrt{2} m_e^{3/2}}{\pi^2 \hbar^3} \sqrt{E} dE
  \Rightarrow
  g(E) = \dfrac{d\tilde n_E}{dE} = \dfrac{\sqrt{2} m_e^{3/2}}{\pi^2 \hbar^3} \sqrt{E}
\]

Распределение свободных электронов по энергиям:
\[
  dn_E = f_\text{Ф-Д} (E) d\tilde N_E = \dfrac{\sqrt{2} m_e^{3/2}}{\pi^2 \hbar^3} \dfrac{\sqrt{E}}{e^{\dfrac{E-E_F}{kT}} + 1}
\]

Можно чуть проанализировать эту формулу, например, в предельном случае $T \to 0$, распределение
Ферми-Дирака представляет собой просто ступеньку, а значит для энергий $E > E_F$ получим 0, то
есть нет таких электронов, а все состояния $E < E_F$ заполнены.

\begin{figure}[H]
		\centering
		\includegraphics[width=.9\linewidth]{img/oral-04/oral-04-distribution-of-free-electrons.png}
		\caption{Графики распределения свободных электронов.
    График распределения при $T \to 0$ (слева-сверху).
    График распределения при увеличении температуры (справа).
    Графическое описания нахождения размытия уровня Ферми (слева-снизу)}
		\label{fig:free-electrons}
	\end{figure}

\subsubsection{Оценка величины "размытия уровня Ферми" при $0 < kT \ll E_F$}

Продолжением анализа прошлого результата (про распределение свободных электронов) является анализ
того, что происходит с распределением при увеличении температуры. Так как при выполненом соотношении
$0 < kT \ll E_F$ график слабо отличается от графика при $T \to 0$ вне некоторого интервала
шириной $\Delta E$ вокруг точки $E_F$, ширину этого интервала можно оценить графически:
провести касательную к графику в точке $E_F$ (учитывая, что $dn_E / dE (E_F) = 1/2$):
\[
  |\tg (\pi - \alpha)| = \dfrac{1/2}{\Delta E} = \dfrac{1}{2 \Delta E}, 
  |\tg (\pi - \alpha)| = |\tg \alpha| = \left( \dfrac{d f_\text{Ф-Д}}{dE} \right)_{E = E_F} = \dfrac{1}{4kT}
  \Rightarrow \Delta E \approx 2kT
\]


\subsubsection{Влияние напряженности электрического поля на процесс ТЭЭ}

\subsubsection{Особенности зонной структуры легированных полупроводников. Оценка величины "энергии ионизации" примесных уровней}

\subsubsection{Строение p-n перехода. Распределение объемного заряда, концетраций дырок и электронов, электрического потенциала. Вывод формулы Шокли}

$p-n$ нереходом называют тонкий слой, образующийся в месте контакта двух областей полупроводников
акцепторного и донорного типов. С одной стороны от перехода будет область $p$ типа, в которой
основным носителем заряда будут дырки, с другой стороны -- область $n$ типа, в которой основными
носителями заряда будут электроны. В области контакта полупроводников носители заряда будут
проходить через переход, вызывая <<рекомбинацию>> -- то есть взаимное <<уничтожение>> электрона и дырки.
За счёт этого вблизи границы концентрация обоих носителей близка к нулю.

Из-за распределения зарядов примесных ионов в зоне контакта возникает контактное электрическое поле,
которое "противодействует" переходу основных носителей через границу контакта. Это приводит к возникновению потенциального барьера высоты $U_0$, кроме того, подключённый источник гапряжения изменяет высоту потенциального барьера: повышает, если минус источника подключён к p-области.

В равновесном состояниии $j_\text{ОСН} = j_\text{НЕОСН}$, $j_\text{ОСН} = j_0 \cdot e^{-\dfrac{U_0}{kT}}$, а полный ток $j = j_\text{ОСН} - j_\text{НЕОСН} = 0$.

Если прикладывается напряжение $V>0$, то $j_\text{ОСН} = j_0 \cdot e^{-\dfrac{U_0-eV}{kT}}$, следовательно, $j = j_\text{ОСН} - j_\text{НЕОСН} = j_0 e^{-\dfrac{U_0-eV}{kT}} - j_0 \cdot e^{-\dfrac{U_0}{kT}} = j(T) \cdot \left( e^{\dfrac{eV}{kT}} - 1 \right) $ - формула Шокли.

\subsubsection{Эффект Холла в полупроводнике с двумя типами носителей заряда}

 \newpage
  \subsection{Ключевые эксперименты}

\subsubsection{Опыт Резерфорда по рассеянию $\alpha$-частиц}

\subsubsection{Измерение спектра излучения АЧТ}

\subsubsection{Фотоэффект}

\subsubsection{Комптон-эффект}
\begin{figure}[H]
	\begin{center}
		\begin{tikzpicture}[thick]
			\begin{scope}
				\path [draw=blue,snake arrow,->]
				(-4,0) -- (-2,0) node[anchor=south]{\hspace{-2cm}$h\nu$};
				\filldraw[black] (-1,0) circle (2pt) node[anchor=north]{$\vec e$};
			\end{scope}
			$\implies$
			\begin{scope}
				\path [draw=blue,dashed] (1,0)--(4,0);
				\path [draw=blue,snake arrow,->]
				(2,0) -- (3.5,1) node[anchor=east]{\hspace{-2cm}$h\nu'$};
				\path [draw=black, ->]
				(2,0) -- (2.5,-1) node[anchor=west]{$\vec p_e$};
				 \draw[black] (3,0) arc (0:31:1) node[anchor=west]{$\theta$};
			\end{scope}
		\end{tikzpicture}
	\end{center}
	\caption{Комптон-эффект}
\end{figure}
\begin{figure}[H]
	\begin{center}
		\begin{tikzpicture}[thick]
			\begin{scope}
				\path[draw=black,->] (-2,0)--(-0.05,0) node[anchor=north]{\hspace{-2cm}$\vec p$};
				\path[draw=black,->] (0,0)--(1,1) node[anchor=west]{\hspace{-2cm}$\vec {p'}$};
				\path[draw=black,->] (0,0)--(1,-1) node[anchor=west]{\hspace{-2cm}$\vec p_e$};
				\draw[black] (1,0) arc (0:45:1) node[anchor=north]{$\theta$};
				\path[draw=black,dashed] (0,0)--(1.4,0) ;
			\end{scope}
%			$\Leftrightarrow$
			\begin{scope}
				\path[draw=black,->] (3,0)--(5,0) node[anchor=north]{\hspace{-2cm}$\vec p$};
				\path[draw=black,->] (3,0)--(4,1) node[anchor=west]{\hspace{-2cm}$\vec {p'}$};
				\path[draw=black,->] (4,1)--(4.95,0.05) node[anchor=south]{\hspace{0.5cm}$\vec p_e$};
				\draw[black] (4,0) arc (0:45:1) node[anchor=north]{$\theta$};
			\end{scope}
		\end{tikzpicture}
	\end{center}
	\caption{Закон сохранения импульса}
\end{figure}


\subsubsection{Дифракция электронов (в чем отличие от дифракции рентгеновских лучей?)}
\begin{itemize}
  \item опыт Девисона-Джермера (монокристалл);
  \item Опыты Томсона  и Тартаковского (поликристалл);
  \item Опыт Фабриканта (одиночные электроны);
\end{itemize}

\subsubsection{Опыт Штерна-Герлаха}
 \newpage
  \subsection{Атомы, молекулы, строение вещества}

\subsubsection{Состав атома (электроны, ядра, нуклоны). Характеристики входящих в него частиц}

Атом состоит из положительного заряженного \textit{ядра} и окружающей его \textit{электронной оболочки}. Линейные размеры ядра порядка $10^{-13}$--$10^{-12}$ см. Размеры самого атома, определяемые электронной оболочкой, примерно в $10^{5}$ раз больше. Почти вся масса атома сосредоточена в ядре. 

Ядро стстоит из <<тяжелых>> протонов и нейтронов (их вместе называют \textit{нуклоны}), а электронная оболочка --- из <<легких>> электронов ($m_p = 1836.15 m_e, m_n = 1838.68 m_e$). Число электронов в оболочке нейтрального атома равно заряду ядра, если за единицу принять элементарный заряд. Но электронная оболочка может терять или приобретать электроны. Тогда атом становится электрически заряженным, т.е. превращается в положительный или отрицательный \textit{ион}.

Химические свойства атома определяются электронной оболочкой, точнее ее наружными электронами. Такие электроны слабо связаны с атомом и поэтому наиболее подвержены электрическим воздействиям со стороны наружных электронов соседних атомов. Наротив, протоны и нейтроны прочно связаны внутри ядра. Однако строение и свойства электронной оболочки определяются электрическим полем ядра атома. 

Протон обладает электрическим зарядом $+e$, имеет полуцелый спин $S_p = \frac{1}{2}$ и собственный магнитный момент.

Нейтрон имеет нулевой заряд и, также, полуцелый спин $S_n = \frac{1}{2}$ и собственный магнитный момент. 

Электрон имеет отрицательный заряд $-1,6022 \times 10^{-19}$ Кл. Полуцелый спин $S_e = \frac{1}{2}$ и массу $m_e = 0.511$ МэВ ($9.1094 \times 10^{-31}$ кг).

\subsubsection{Магнитный орбитальный момент. Магнетон Бора $\mu_B$ (вывод)}

Если рассматривать атом водорода с квантовым числом $n=0$, что говорит нам о том, что квантовое
число $m \in \left\{ 0 \right\} \Leftrightarrow m=0$, то есть $L_z = 0$. С точки зрения
классической механики, электрон движется
по круговой орбите радиуса $r$ с круговой частотой $\omega$. Сила тока: $I = \dfrac{dq}{dT}$.
За время $T = \dfrac{2\pi}{\omega}$ электрон полностью пройдёт по орбите, а значит $dq = e$, 
$I = \dfrac{e \omega}{2\pi}$. Магнитный момент, по определению, равен
$\vec{P}_m = \dfrac{I}{2} \oint [\vec{r}, \vec{dl}] = I S \vec{n}$, где S -- площадь
поверхности, замкнутой внутри контура, $\vec{n}$ -- нормаль к этой поверхности.
Получается, что $P_m = \dfrac{e}{T} \pi r^2 = \dfrac{e \omega r^2}{2}$. Момент импульса:
$\vec{L_z} = \vec{r} \times \vec{p}$,
$L = m_e vr = m_e \omega r^2$.
Гиромагнитным соотношением называется $\gamma = \dfrac{P_m}{L_z} = \dfrac{e}{2m_e}$. 
То есть $P_m = \gamma L_z = \dfrac{e \hbar}{2 m_e} m = \mu_\text{Б} m$, где $m$ -- орбитальное квантовое число.
Постоянная $\mu_B = \dfrac{e \hbar}{2 m_e}$ называется магнетоном Бора. Как видно, это 
своеобразная <<единица измерения>> магнитного момента. 

Но однако из эксперимента Штерна и Герлаха видно, что для атомов серебра с $m=0$ всё равно
почему-то магнитный момент ненулевой. Из этого Уленбек и Гаудемит выдвинули гипотезу о том, 
что у электрона есть свой собсвтенный механический момент, не связанный с его движением как целого,
то есть с движение его центра масс. Этот собственный механический момент называют спином.
А момент импульса, который добавляется -- $L_z^s$ называют спиновым механическим моментом.

Куча терминов:
\begin{itemize}
  \item $s$ -- спиновое квантовое число;
  \item $s_m$ -- пиновое магнитное квантовое число;
  \item $L_z^L, L_z^s$ -- орбитальный и спиновые моменты;
  \item $P_{mz}^s$ -- спиновый магнитный момент.
    Для него верно: $P_{mz}^s = 2 \mu_B \cdot s_m \in \left\{ -\mu_B, \mu_B \right\}$.
    Этот факт был получен экспериментально;
  \item $L_z = L_z^L + L_z^s$ -- полный;
  \item $L_z^L = \hbar m, m \in \left\{ 0, \pm 1, \pm 2, \dots, \pm l \right\}, l \in \left\{ 0, 1, \dots, n-1 \right\}$ -- орбитальный момент;
  \item $L_z^s = \hbar m_s, m_s \in \left\{ - \dfrac{1}{2}, \dfrac{1}{2} \right\}$ -- спиновый момент.
\end{itemize}

\end{document}
