\subsection{Фундаментальные понятия и принципы Квантовой Механики}

\subsubsection{Волны Де Бройля}

\paragraph{Общий случай} Плоская волна частотой $\omega$, распространяющаяся вдоль оси $x$:
\begin{equation*}
	\xi(x,t) = A \exp \left\{-i(\omega t -kx)\right\},
\end{equation*}
где $A$ - амплитуда, $k=\frac{2\pi}{\lambda}$ - волновое число.

\paragraph{Гипотеза Де Бройля} 
Свободной частице с энергией $E$ и импульсом $p$, движущейся вдоль $x$, соответсвует плоская волна:
\begin{equation*}
	\Psi(x, t) = A \exp \left\{-\frac{i}{\hbar}(E t - px)\right\}
\end{equation*}
Посмотрев на выражения волновых процессов в общем виде $\xi$ и для волны Де Бройля $\Psi$, заметим некоторые соотношения:
\begin{itemize}
	\item частота $\omega = \frac{E}{\hbar}$
	\item длина волны Де Бройля $\lambda_{\text{Б}}=\frac{2\pi\hbar}{p}$
	\item энергия $E = \hbar \omega = h\nu$
	\item $\vec p = h \vec k$, где $\vec k$ - волновой вектор
\end{itemize}

\paragraph{Свойства волн Де Бройля}
\begin{enumerate}
	\item В процессе распространения волны могут отражаться, преломляться, интерферировать и дифрагировать по обычным волновым законам
	
	\item Фазовая скорость $v_\text{фаз}$ - скорость, с которой распространяются точки волны с постоянной фазой.
	Выражение для фазовой скорости вытекает из условия постоянности фазы при его дифференцировании:
	\begin{equation*}
		Et-px=\mathrm{const} \overset{\dfrac{d}{dt}}{\longrightarrow} v_\text{фаз} = \frac{dx}{dt} = \frac{E}{p}.
	\end{equation*}
	При подстановке известных соотношений, а именно $E=mc^2$ и $p=mv$ в $v_\text{фаз}$, получим
	\begin{equation*}
		v_\text{фаз} = \frac{c^2}{v}
	\end{equation*} выражение для фазовой скорости\footnote{$v<c \implies v_\text{фаз} = \frac{c^2}{v} > c$ --- это не противоречит СТО. Ограничения, накладываемые СТО касаются скорости переноса массы/энергии, но фазовая скорость волны ничего из этого не характеризует.}.
	
	\item Групповая скорость $v_\text{гр}$
	\begin{equation*}
		v_\text{гр} = \frac{dw}{dk} = \frac{d(\hbar\omega)}{d(\hbar k)} = \frac{dE}{dp}
	\end{equation*}
	Возьмем известное всем выражение из теории относительности:
	\begin{equation*}
		E^2=p^2c^2+E_0^2=\left[E_0=m_0c^2\right]=E^2=p^2c^2+m_0^2c^4\overset{\dfrac{d}{dt}}{\longrightarrow}2EdE=2pc^2dp\longrightarrow\frac{dE}{dp}=\frac{pc^2}{E}
	\end{equation*}
	Воспользовавшись этим знанием, получаем, что групповая скорость волны равна скорости движения частицы:
	\begin{equation*}
		v_\text{гр} = \frac{pc^2}{E} = \frac{pc^2}{mc^2} = \frac{p}{m} = v
	\end{equation*}
	
	\item Длина волны Де Бройля для нерелятивистских $(v\ll c)$ и релятивистких ($v\approx c$) частиц.
	\begin{itemize}
		\item Нерелятивисткий случай $(v\ll c)$:
		\begin{equation*}
		E_k=\frac{mv^2}{2}=\frac{p^2}{2m_0}\implies p=\sqrt{2m_0E_k}\implies \lambda_\text{Б}=\frac{2\pi\hbar}{p}=\frac{2\pi\hbar}{\sqrt{2m_0E_k}}
		\end{equation*}
		\item Релятивистский случай $(v\approx c)$:
		\begin{equation*}
		p=\frac{1}{c}\sqrt{E_k(E_k+2m_0c^2)}=\sqrt{2m_0E_k}\sqrt{1+\frac{E_k}{2m_0c^2}}\implies\lambda_\text{Б}'=\frac{2\pi\hbar}{\sqrt{2m_0E_k}\sqrt{1+\frac{E_k}{2m_0c^2}}}
		\end{equation*}
		В этом выражении можно выделить нерелятивистскую дебройлевскую длину волны $\lambda_\text{Б}$, тогда 
		\begin{equation*}
			\lambda_\text{Б}'=\frac{\lambda_\text{Б}}{\sqrt{1+\frac{E_k}{2m_0c^2}}}
		\end{equation*}
	\end{itemize}
\end{enumerate}



\subsubsection{Соотношения неопределенностей Гейзенберга}

\subsubsection{Постулаты квантовой механики}

\subsubsection{Принцип суперпозиции квантовых состояний}

\subsubsection{Принцип неразличимости тождественных частиц}
