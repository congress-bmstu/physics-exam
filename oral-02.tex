\subsection{Следствия из основных положений}

\subsubsection{Уравнение непрерывности для плотности вероятности, как следствие основных
постулатов и уравнения Шрёдингера (вывод)}

Уравнение Шредингера:
\[
  -i \hbar \dfrac{\partial \Psi}{\partial t} = - \dfrac{\hbar^2}{2m} \nabla^2 \Psi + U \Psi,
\]
комплексно сопряженное уравнение:
\[
  -i \hbar \dfrac{\partial \Psi^*}{\partial t} = - \dfrac{\hbar^2}{2m} \nabla^2 \Psi^* + U \Psi^*,
\]

Умножим обычное на $\Psi^*$, а компескно сопряженное умножим на $\Psi$, потом сложим:
\[
  i \hbar \left( \Psi^* \dfrac{\partial \Psi}{\partial t} + \Psi \dfrac{\partial \Psi^*}{\partial t} \right)
  = - \dfrac{\hbar^2}{2m} \left( \Psi^* \nabla^2 \Psi - \Psi \nabla^2 \Psi^* \right) 
  \Rightarrow
  \left( \Psi^* \dfrac{\partial \Psi}{\partial t} + \Psi \dfrac{\partial \Psi^*}{\partial t} \right)
  = \dfrac{i \hbar}{2m} \left( \Psi^* \nabla^2 \Psi - \Psi \nabla^2 \Psi^* \right) 
\]
левая часть равенства представляет собой $\dfrac{\partial \Psi^* \Psi}{\partial t} =
\dfrac{\partial |\Psi|^2}{\partial t}$, то есть производную плотности вероятности $\rho_p$
по времени. В правой части преобразуем:
\[
  \left( \Psi^* \nabla^2 \Psi - \Psi \nabla^2 \Psi^* \right)
  = \nabla \left( \Psi^* \nabla \Psi - \Psi \nabla \Psi^* \right)
\]
(это выражение проверяется в лоб). Обозначая $ - \dfrac{i \hbar}{2m} \left( \Psi^* \nabla \Psi - \Psi \nabla \Psi^* \right) = \vec{J}_p$, окончательно получаем:
\[
  \dfrac{\partial \rho_p}{\partial t} = - \nabla \vec{J}_p =  - \operatorname{div} \vec{J}_p.
\]
Это выражение называется уравнением непрерывности для плотности вероятности.

\subsubsection{Симметричные и антисимметричные состояния}

Для системы нескольких частиц, волновая функция представляется в виде: $\Phi( \vec{q}_1, \vec{q}_2, \dots, \vec{q}_n )$. Рассмотрим оператор перестановки $\hat{P}_{ij}$:
\[
  \hat{P}_{ij} \Psi(\vec{q}_1, \dots, \vec{q}_i, \dots, \vec{q}_j, \dots, \vec{q}_n)
  = \Psi(\vec{q}_1, \dots, \vec{q}_j, \dots, \vec{q}_i, \dots, \vec{q}_n)
\]
рассмотрим спектр этого оператора:
\[
  \hat{P}_{ij} \Psi = \lambda \Psi, \hat{P}^2_{ij} \Psi = \Psi = \lambda^2 \Psi 
  \Rightarrow
  \lambda = \pm 1.
\]
Функции $\Psi$, у которых $\lambda = 1$, называются симметричными, а другие -- антисимметричными. Причём несложно доказать, что если системе одинаковых частиц в некоторый момент времени соответствовала функция (анти) симметричная, то и в последующие моменты времени она останется такой же:
\[
  - i \hbar \dfrac{\partial \Psi}{\partial t} = \bar{H} \Psi
\]
ну и вот справа будет функция такой же симметричности как и $\Psi$, поэтому и слева тоже.

\subsubsection{Принцип Паули в строгой формулировке}



\subsubsection{Нахождение среднего значения физической величины (вывод)}

Пусть некоторой физической величине $f$ соотвествует оператор $\hat{f}$. Найдём его собственные значения и обозначим их $f_n$, а собственные функции -- $\Psi_n$.

Тогда любое состояние $\Psi$ раскладывается по базису из собственных функций этого оператора:
\[
  \Psi = \sum_n c_n \Psi_n
\]
это утверждение носит название 3-его постулата КМ. Причём вероятность $P(f = f_n)$ равняется квадрату модуля коэффициента перед соотвествующей собственной функцией:
\[
  P(f = f_n) = P_n = c_n^* c_n.
\]

Тогда среднее значение (матожидание) некоторой величины $f$ выражается:
\begin{multline*}
  <f> = \sum_n P_n f_n
  = \sum_n c_n^* c_n f_n
  = \sum_n c_n^* f_n \int \Psi_n^* \Psi dV
  = \sum_n c_n^* \int \Psi \hat{f} \Psi_n^*
  = \int \Psi (\hat{f} \sum_n c_n \Psi_n)^* dV = \\
  = \int \Psi (\hat{f} \Psi)^* dV
  = \int \Psi^* (\hat{f} \Psi) dV
\end{multline*}

\subsubsection{Дискретность спектра энергий для связанных состояний квантовых систем
(Объяснить связь с основными постулатами)}

Связным состоянием называется такое состояние, в котором частица движется в ограниченной области
пространства $\Omega$, следовательно, граничное условие для $\Psi$:
$\left. \Psi \right|_{\partial \Omega} = 0$. В таком случае спектр энергий квантуется.

Из ландафщица:

Спектр собственных значений энергии может быть как дискретным, так и непрерывным.
Стационарное состояние дискретного спектра всегда соответствует финитному движению
системы, т. е. движению, при котором система или какая-либо ее часть не уходит на
бесконечность. Действительно, для собственных функций дискретного спектра интеграл
$\int |\Phi|^2 dq$, взятый по всему пространству, конечен. Это, во всяком случае,
означает, что квадрат $|\Phi|^2$ достаточно быстро убывает, обращаясь на бесконечности в
нуль. Другими словами, вероятность бесконечных значений координат равна нулю, т. е.
система совершает финитное движение или, как говорят, находится в связанном состоянии.

Для волновых функций непрерывного спектра интеграл $\int | \Phi |^2 dq$ расходится.
Квадрат волновой функции $|\Phi|^2$ не определяет здесь непосредственно вероятности
различных значений координат и должен рассматриваться лишь как величина, пропорциональная
этой вероятности. Расходимость интеграла $\int | \Phi |^2 dq$ всегда бывает связана с тем,
что $|\Phi|^2$ не обращается на бесконечности в нуль (или обращается в нуль недостаточно
быстро). Поэтому можно утверждать, что интеграл $\int | \Phi |^2 dq$, взятый по
области пространства, внешней по отношению к любой сколь угодно большой, но конечной
замкнутой поверхности, будет все же расходиться. Это значит, что в рассматриваемом
состоянии система (или какая-либо ее часть) находится на бесконечности. Для волновой функции,
представляющей собой суперпозицию волновых функций различных стационарных состояний
непрерывного спектра, интеграл $\int |\Phi|^2 dq$ может оказаться сходящимся, так что
система находится в конечной области пространства. Однако с течением времени эта область
будет неограниченно смещаться, и в конце концов система уходит на бесконечность.


\subsubsection{Квантовое туннелирование и его проявление в различных явлениях ($\alpha$-распад, холодная эмиссия, тунельный микроскоп)}

Определение. Туннельный эффект или туннелирование - это явление преодоления микрочастицей
потенциального барьера в случае, когда её полная энергия (остающаяся при туннелировании
неизменной) меньше высоты барьера.

\paragraph{Альфа распад}
Альфа-частица испытывает туннельный переход через потенциальный барьер, обусловленный
ядерными силами, поэтому альфа-распад является существенно квантовым процессом.

\paragraph{Холодная эмиссия}


\paragraph{Тунельный микроскоп}
Рассматривать отдельные атомы можно с помощью устройства, использующего квантовый эффект
туннелирования – сканирующий туннельный микро- скоп (СТМ). Точнее, сканирующий туннельный
микроскоп не рассматривает, а как бы «ощупывает» ис- следуемую поверхность. Очень тонкая
игла-зонд с острием толщиной в один атом перемещается над поверхностью объекта на расстоянии
порядка одного нанометра. При этом согласно законам квантовой механики, электроны преодолевают
вакуумный барьер между объектом и иглой – туннелируют, и между зондом и образцом начинает
течь ток. Сила этого тока очень сильно зависит от расстояния между концом иглы и поверхностью
образца – при изменении зазора на десятые доли нанометра сила тока может возрасти или
уменьшиться на порядок. Так что, перемещая зонд вдоль поверхности с помощью пьезоэлементов
и отслеживая изменение силы тока, можно исследовать ее рельеф практически «на ощупь».

