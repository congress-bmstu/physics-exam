\subsection{Элементы статистической физики}
\subsubsection{Принцип детального равновесия и его применение (вывод закона Кирхгофа, связь $r_{T, \omega}^*$ vs $u_{T, \omega}$, формула Эйнштейна для коэффициентов A и B)}
\paragraph{Формула Эйнштейна для коэффициентов A и B} см. письменный вопрос \ref{einstein-a-b}

\subsubsection{Плотность квантовых состояний и плотность мод колебаний электромагнитного поля в полости}
\textbf{Плотность квантовых состояний}.
Найдём число квантовых состояний, по которым могут распределяться частицы, при
условии, что энергия этих состояний не превышает некоторого значения $ E $.
Пусть сначала частица находится в трёхмерной потенциальной яме с непроницаемыми
стенками. Энергия в такой яме равна 
\[
  E = \frac{\pi^2\hbar^2}{2m_0} \left[ \left( \frac{n_1}{a_1} \right)^2 + \left(
  \frac{n_2}{a_2}\right)^2 + \left( \frac{n_3}{a_3} \right)^2   \right],
\]
где $ a_1 $, $ a_2 $ и $ a_3 $ --- стороны прямоугольного параллелепипеда, а $
n_1 $, $ n_2 $, $ n_3 = 1, 2, 3,\ldots$ --- квантовые числа. Пусть $ E $ столь
велико, что спектр можно считать практически непрерывным.

В дискретном трёхмерном пространстве квантовых чисел введём обозначение  
\[
  r^2 = \frac{(n_1a_2a_3)^2 + (n_2a_1a_3)^2 + (n_3a_1a_2)^2}{(a_1a_2a_3)^{4/3}}
\]
и перепишем выражение для энергии в виде 
\[
  E = \frac{\pi^2\hbar^2}{2m_0(a_1a_2a_3)^{2/3}}r^2,
\]
откуда 
\[
  r = \frac{(a_1a_2a_3)^{1/3}\sqrt{2m_0E}}{\pi\hbar}.
\]

Теперь коль скоро $ n_1^2 + n_2^2 + n_3^2 < r^2 $, то и $ E < E_{\max} $ (?). Рассмотрев теперь сферу (её восьмую часть) с радиусом $ r $, найдём число
возможных состояний как объём этой фигуры:
\[
  G = \frac{1}{6}\pi r^3 J_z = \frac{1}{6} \pi \frac{ \left( \sqrt{2m_0E}
  \right)^3 }{\pi^3\hbar^3}J_z a_1a_2a_3,
\]
где $ J_z $ есть количество возможных проекций спина частицы (для электрона,
например, они принимают значения $ \pm 1/2 $, откуда $ J_z = 2 $).
Действительно, каждой точке пространства соответствует $ J_z $ состояний.
Поскольку $ a_1a_2a_3 $ представляет собой объём потенциальной ямы $ V $, а $
\sqrt{2m_0E} $ есть нерелятивистский импульс частицы $ p $, то соотношение можно
переписать в виде 
\[
  G = \frac{4}{3} \pi p^3 V \frac{J_z}{(2\pi\hbar)^3}.
\]
При этом $ (4\pi p^3)/3 $ есть не что иное, как объём шара радиусом $ p $
--- импульса, соответствущего максимальной энергии $ E $ --- в пространстве
импульсов $ p_x $, $ p_y $, $ p_z $. Таким образом, выражение для $ G $ в
фазовом пространстве $ x $, $ y $, $ z $, $ p_x $, $ p_y $, $ p_z $ имеет вид 
\[
  G = \frac{V_{\text{фаз}}}{(2\pi \hbar)^3} J_z,
\]
то есть $ G $ пропорционально фазовому объёму.

Оказывается, что данный результат справедлив для ям произвольной формы.

Таким образом, объём фазового пространства, приходящийся на одно квантовое
состояние, равен $ (2\pi\hbar)^3 = h^3 $. Запишем это следующим образом: 
\[
  \Delta x \Delta y \Delta z \Delta p_x \Delta p_y \Delta p_z = h^3,
\]
где $ \Delta x, \ldots \Delta p_x, \ldots $ --- размеры ячейки в фазовом пространстве, приходящейся
на одно состояние. При этом благодаря равноправности координат (?), например, $
\Delta x \Delta
p_x = 2\pi \hbar$. Этот результат, как легко видеть, согласуется с принципом 
неопределенности. Действительно, размеры ячейки фазового 
пространства, приходящейся на одно состояние, должны определяться
теми ограничениями на значения координаты и импульса, которые
накладывают соотношения неопределенностей.

Найдём теперь \emph{плотность квантовых состояний} $ g(E) $, то есть число
состояний, приходящихся на единичный интервал энергий. Найдём его, переписав в
виде 
\[
  g(E)  = \frac{dG}{dp} \frac{dp}{dE} = J_z \frac{dp}{dE} \frac{d}{dp} \left(
  \frac{4}{3} \frac{\pi p^3 V}{(2\pi \hbar)^3}\right) = J_z \frac{4\pi p^2
  V}{(2\pi\hbar)^3} \frac{dp}{dE}.
\]
Данное выражение является общим, то есть справедливым для любых частиц.

\textbf{Плотность мод колебаний электромагнитного поля в полости}.
\emph{Модами}, или \emph{собственными колебаниями} называют набор характерных
для колебательной системы типов гармонических колебаний.


\subsubsection{Характерные свойства теплового излучения в полости, как следствие его равновесности}
Введём ряд допущений. Пусть вещество состоит из одинаковых не взаимодействующих друг с другом атомов,
которые могут находиться в двух квантовых состояниях --- с энергией $ E_1 $
(основное) и $
E_2 > E_1$ (возбуждённое) соответственно. Отбросим все причины возбуждения,
кроме поглощения атомом излучения с частотой $ \omega $, удовлетворяющей
квантовому условию  
\[
    \hbar \omega = E_2 - E_1.
\]
Тогда излучение в полости будет монохроматическим (малый разброс частот) с
частотой $ \omega $. Пусть $ N_1 $, $ N_2 $ --- число атомов с соответствующей
энергией, всего их $ N := N_1 + N_2 $. По \emph{формуле распределения
Больцмана} 
\[
    \frac{N_2}{N_1} = \exp \left( - \frac{E_2 - E_1}{kT} \right),
\]
где $ k $ --- постоянная Больцмана, $ T $ --- температура. Видно, что для
равновесной системы $ N_2 < N_1 $, однако при $ T \to \infty $ $ N_1 \to N_2 $.




