\subsection{Элементы физики твердого тела}

\subsubsection{Трансляционная симметрия --  определение, примеры объектов с трансляционной симметрией}

\subsubsection{Функция распределения для свободных электронов (вывод). Связь энергии Ферми и концентрации электронов}

Будем рассматривать металл как потенциальный ящик с абсолютно непроницаемыми стенками, в 
котором находятся электроны, не взаимодействующие между собой. Рассмотрим число квантовых состояний 
$N_E$, энергия которых меньше $E$ ($l_i$ -- размер коробки по $i$-ой координате):
\[
  E_{n_1, n_2, n_3} = \dfrac{\pi^2 \hbar^2}{2 m_e} \left(
    \left( \dfrac{n_1}{l_1} \right)^2 +
    \left( \dfrac{n_2}{l_2} \right)^2 +
    \left( \dfrac{n_3}{l_3} \right)^2 \right) < E
\]
обозначая: $V = l_1 l_2 l_3 = l_0^3$, $\alpha_i = \dfrac{l_i}{l_0}$, получаем:
\[
  \dfrac{\pi^2 \hbar^2}{2 m_e l_0^3} \left(
    \left( \dfrac{n_1}{\alpha_1} \right)^2 +
    \left( \dfrac{n_2}{\alpha_2} \right)^2 +
    \left( \dfrac{n_3}{\alpha_3} \right)^2 \right) < E
\]
обозначим $E_0 = \dfrac{\pi^2 \hbar^2}{2 m_e l_0^3}$, $\varepsilon = \dfrac{E}{E_0}$:
\[
  \dfrac{1}{\varepsilon} \left(
    \left( \dfrac{n_1}{\alpha_1} \right)^2 +
    \left( \dfrac{n_2}{\alpha_2} \right)^2 +
    \left( \dfrac{n_3}{\alpha_3} \right)^2 < 1
  \Leftrigtharrow
  \left( \dfrac{n_1}{\alpha_1 \sqrt{\varepsilon}} \right)^2 +
    \left( \dfrac{n_2}{\alpha_2 \sqrt{\varepsilon}} \right)^2 +
    \left( \dfrac{n_3}{\alpha_3 \sqrt{\varepsilon}} \right)^2 < 1
\]
Таким образом, приходим к тому, что нужно найти объём в пространстве квантовых чисел
(квазинепрерывном), это уравнение эллипса, поэтому объем равен 1/8 части объёма эллипса
(учитывая то, что каждое кв число больше 0):
\[
  \Omega_E = \dfrac{1}{8} \dfrac{4}{3} \pi (\sqrt{\varepsilon})^3 \alpha_1 \alpha_2 \alpha_3
  = \dfrac{1}{6} \pi \varepsilon^{3/2}.
  \Rightarrow
  N_\text{Я} = \dfrac{1}{6} \pi \varepsilon^{3/2}
  = \dfrac{1}{6} \pi E^{3/2} \left( \dfrac{2 m_e l_0^3}{3 \pi^2 \hbar^2} \right)^{3/2}
  = \dfrac{\sqrt{2} m_e^{3/2}}{3 \pi^2 \hbar^3} E^{3/2} V
\]
с учётом спина:
\[
  N_E = 2 N_\text{Я} = \dfrac{2 \sqrt{2} m_e^{3/2}}{3 \pi^2 \hbar^3} E^{3/2} V 
\]
перейдём к концентрации:
\[
  \tilde n_E = \dfrac{N_E}{V} = \dfrac{2 \sqrt{2} m_e^{3/2}}{3 \pi^2 \hbar^3} E^{3/2}.
\]
Это количество состояний, для которых энергия меньше $E$, для получения концентрации состояний в
отрезке $[E, E+\Delta E]$, продифференцируем это:
\[
  dn_E = \dfrac{\sqrt{2} m_e^{3/2}}{\pi^2 \hbar^3} \sqrt{E} dE
  \Rightarrow
  g(E) = \dfrac{d\tilde n_E}{dE} = \dfrac{\sqrt{2} m_e^{3/2}}{\pi^2 \hbar^3} \sqrt{E}
\]

Распределение свободных электронов по энергиям:
\[
  dn_E = f_\text{Ф-Д} (E) d\tilde N_E = \dfrac{\sqrt{2} m_e^{3/2}}{\pi^2 \hbar^3} \dfrac{\sqrt{E}}{e^{\dfrac{E-E_F}{kT}} + 1}
\]

Можно чуть проанализировать эту формулу, например, в предельном случае $T \to 0$, распределение
Ферми-Дирака представляет собой просто ступеньку, а значит для энергий $E > E_F$ получим 0, то
есть нет таких электронов, а все состояния $E < E_F$ заполнены.

\begin{figure}[H]
		\centering
		\includegraphics[width=.9\linewidth]{img/oral-04/oral-04-distribution-of-free-electrons.png}
		\caption{Графики распределения свободных электронов.
    График распределения при $T \to 0$ (слева-сверху).
    График распределения при увеличении температуры (справа).
    Графическое описания нахождения размытия уровня Ферми (слева-снизу)}
		\label{fig:free-electrons}
	\end{figure}

\subsubsection{Оценка величины "размытия уровня Ферми" при $0 < kT \ll E_F$}

Продолжением анализа прошлого результата (про распределение свободных электронов) является анализ
того, что происходит с распределением при увеличении температуры. Так как при выполненом соотношении
$0 < kT \ll E_F$ график слабо отличается от графика при $T \to 0$ вне некоторого интервала
шириной $\Delta E$ вокруг точки $E_F$, ширину этого интервала можно оценить графически:
провести касательную к графику в точке $E_F$ (учитывая, что $dn_E / dE (E_F) = 1/2$):
\[
  |\tg (\pi - \alpha)| = \dfrac{1/2}{\Delta E} = \dfrac{1}{2 \Delta E}, 
  |\tg (\pi - \alpha)| = |\tg \alpha| = \left( \dfrac{d f_\text{Ф-Д}}{dE} \right)_{E = E_F} = \dfrac{1}{4kT}
  \Rightarrow \Delta E \approx 2kT
\]


\subsubsection{Влияние напряженности электрического поля на процесс ТЭЭ}

\subsubsection{Особенности зонной структуры легированных полупроводников. Оценка величины "энергии ионизации" примесных уровней}

\subsubsection{Строение p-n перехода. Распределение объемного заряда, концетраций дырок и электронов, электрического потенциала. Вывод формулы Шокли}

$p-n$ нереходом называют тонкий слой, образующийся в месте контакта двух областей полупроводников
акцепторного и донорного типов. С одной стороны от перехода будет область $p$ типа, в которой
основным носителем заряда будут дырки, с другой стороны -- область $n$ типа, в которой основными
носителями заряда будут электроны. В области контакта полупроводников носители заряда будут
проходить через переход, вызывая <<рекомбинацию>> -- то есть взаимное <<уничтожение>> электрона и дырки.
За счёт этого вблизи границы концентрация обоих носителей близка к нулю.

Из-за распределения зарядов примесных ионов в зоне контакта возникает контактное электрическое поле,
которое "противодействует" переходу основных носителей через границу контакта. Это приводит к возникновению потенциального барьера высоты $U_0$, кроме того, подключённый источник гапряжения изменяет высоту потенциального барьера: повышает, если минус источника подключён к p-области.

В равновесном состояниии $j_\text{ОСН} = j_\text{НЕОСН}$, $j_\text{ОСН} = j_0 \cdot e^{-\dfrac{U_0}{kT}}$, а полный ток $j = j_\text{ОСН} - j_\text{НЕОСН} = 0$.

Если прикладывается напряжение $V>0$, то $j_\text{ОСН} = j_0 \cdot e^{-\dfrac{U_0-eV}{kT}}$, следовательно, $j = j_\text{ОСН} - j_\text{НЕОСН} = j_0 e^{-\dfrac{U_0-eV}{kT}} - j_0 \cdot e^{-\dfrac{U_0}{kT}} = j(T) \cdot \left( e^{\dfrac{eV}{kT}} - 1 \right) $ - формула Шокли.

\subsubsection{Эффект Холла в полупроводнике с двумя типами носителей заряда}

