\subsection{Элементы физики твердого тела}

\subsubsection{Трансляционная симметрия --  определение, примеры объектов с трансляционной симметрией}

\subsubsection{Функция распределения для свободных электронов (вывод). Связь энергии Ферми и концентрации электронов}



\subsubsection{Оценка величины "размытия уровня Ферми" при $0 < kT \ll E_F$}

\subsubsection{Влияние напряженности электрического поля на процесс ТЭЭ}

\subsubsection{Особенности зонной структуры легированных полупроводников. Оценка величины "энергии ионизации" примесных уровней}

\subsubsection{Строение p-n перехода. Распределение объемного заряда, концетраций дырок и электронов, электрического потенциала. Вывод формулы Шокли}

$p-n$ нереходом называют тонкий слой, образующийся в месте контакта двух областей полупроводников
акцепторного и донорного типов. С одной стороны от перехода будет область $p$ типа, в которой
основным носителем заряда будут дырки, с другой стороны -- область $n$ типа, в которой основными
носителями заряда будут электроны. В области контакта полупроводников носители заряда будут
проходить через переход, вызывая <<рекомбинацию>> -- то есть взаимное <<уничтожение>> электрона и дырки.
За счёт этого вблизи границы концентрация обоих носителей близка к нулю.

Из-за распределения зарядов примесных ионов в зоне контакта возникает контактное электрическое поле,
которое "противодействует" переходу основных носителей через границу контакта. Это приводит к возникновению потенциального барьера высоты $U_0$, кроме того, подключённый источник гапряжения изменяет высоту потенциального барьера: повышает, если минус источника подключён к p-области.

В равновесном состояниии $j_\text{ОСН} = j_\text{НЕОСН}$, $j_\text{ОСН} = j_0 \cdot e^{-\dfrac{U_0}{kT}}$, а полный ток $j = j_\text{ОСН} - j_\text{НЕОСН} = 0$.

Если прикладывается напряжение $V>0$, то $j_\text{ОСН} = j_0 \cdot e^{-\dfrac{U_0-eV}{kT}}$, следовательно, $j = j_\text{ОСН} - j_\text{НЕОСН} = j_0 e^{-\dfrac{U_0-eV}{kT}} - j_0 \cdot e^{-\dfrac{U_0}{kT}} = j(T) \cdot \left( e^{\dfrac{eV}{kT}} - 1 \right) $ - формула Шокли.

\subsubsection{Эффект Холла в полупроводнике с двумя типами носителей заряда}

