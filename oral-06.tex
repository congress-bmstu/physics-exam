\subsection{Атомы, молекулы, строение вещества}

\subsubsection{Состав атома (электроны, ядра, нуклоны). Характеристики входящих в него частиц}

Атом состоит из положительного заряженного \textit{ядра} и окружающей его \textit{электронной оболочки}. Линейные размеры ядра порядка $10^{-13}$--$10^{-12}$ см. Размеры самого атома, определяемые электронной оболочкой, примерно в $10^{5}$ раз больше. Почти вся масса атома сосредоточена в ядре. 

Ядро стстоит из <<тяжелых>> протонов и нейтронов (их вместе называют \textit{нуклоны}), а электронная оболочка --- из <<легких>> электронов ($m_p = 1836.15 m_e, m_n = 1838.68 m_e$). Число электронов в оболочке нейтрального атома равно заряду ядра, если за единицу принять элементарный заряд. Но электронная оболочка может терять или приобретать электроны. Тогда атом становится электрически заряженным, т.е. превращается в положительный или отрицательный \textit{ион}.

Химические свойства атома определяются электронной оболочкой, точнее ее наружными электронами. Такие электроны слабо связаны с атомом и поэтому наиболее подвержены электрическим воздействиям со стороны наружных электронов соседних атомов. Наротив, протоны и нейтроны прочно связаны внутри ядра. Однако строение и свойства электронной оболочки определяются электрическим полем ядра атома. 

Протон обладает электрическим зарядом $+e$, имеет полуцелый спин $S_p = \frac{1}{2}$ и собственный магнитный момент.

Нейтрон имеет нулевой заряд и, также, полуцелый спин $S_n = \frac{1}{2}$ и собственный магнитный момент. 

Электрон имеет отрицательный заряд $-1,6022 \times 10^{-19}$ Кл. Полуцелый спин $S_e = \frac{1}{2}$ и массу $m_e = 0.511$ МэВ ($9.1094 \times 10^{-31}$ кг).

\subsubsection{Магнитный орбитальный момент. Магнетон Бора $\mu_B$ (вывод)}

Если рассматривать атом водорода с квантовым числом $n=0$, что говорит нам о том, что квантовое
число $m \in \left\{ 0 \right\} \Leftrightarrow m=0$, то есть $L_z = 0$. С точки зрения
классической механики, электрон движется
по круговой орбите радиуса $r$ с круговой частотой $\omega$. Сила тока: $I = \dfrac{dq}{dT}$.
За время $T = \dfrac{2\pi}{\omega}$ электрон полностью пройдёт по орбите, а значит $dq = e$, 
$I = \dfrac{e \omega}{2\pi}$. Магнитный момент, по определению, равен
$\vec{P}_m = \dfrac{I}{2} \oint [\vec{r}, \vec{dl}] = I S \vec{n}$, где S -- площадь
поверхности, замкнутой внутри контура, $\vec{n}$ -- нормаль к этой поверхности.
Получается, что $P_m = \dfrac{e}{T} \pi r^2 = \dfrac{e \omega r^2}{2}$. Момент импульса
выражается через магнитный момент: $\vec{L_z} = \vec{r} \times \vec{P}_m$,
$L = m_e vr = m_e \omega r^2$.
Гиромагнитным соотношением называется $\gamma = \dfrac{P_m}{L_z} = \dfrac{e}{2m_e}$. 
То есть $P_m = \gamma L_z = \dfrac{e \hbar}{2 m_e} m = \mu_\text{Б} m$, где $m$ -- орбитальное квантовое число.
Постоянная $\mu_B = \dfrac{e \hbar}{2 m_e}$ называется магнетоном Бора. Как видно, это 
своеобразная <<единица измерения>> магнитного момента. 

Но однако из эксперимента Штерна и Герлаха видно, что для атомов серебра с $m=0$ всё равно
почему-то магнитный момент ненулевой. Из этого Уленбек и Гаудемит выдвинули гипотезу о том, 
что у электрона есть свой собсвтенный механический момент, не связанный с его движением как целого,
то есть с движение его центра масс. Этот собственный механический момент называют спином.
А момент импульса, который добавляется -- $L_z^s$ называют спиновым механическим моментом.

Куча терминов:
\begin{itemize}
  \item $s$ -- спиновое квантовое число;
  \item $s_m$ -- пиновое магнитное квантовое число;
  \item $L_z^L, L_z^s$ -- орбитальный и спиновые моменты;
  \item $P_{mz}^s$ -- спиновый магнитный момент.
    Для него верно: $P_{mz}^s = 2 \mu_B \cdot s_m \in \left\{ -\mu_B, \mu_B \right\}$.
    Этот факт был получен экспериментально;
  \item $L_z = L_z^L + L_z^s$ -- полный;
  \item $L_z^L = \hbar m, m \in \left\{ 0, \pm 1, \pm 2, \dots, \pm l \right\}, l \in \left\{ 0, 1, \dots, n-1 \right\}$ -- орбитальный момент;
  \item $L_z^s = \hbar m_s, m_s \in \left\{ - \dfrac{1}{2}, \dfrac{1}{2} \right\}$ -- спиновый момент.
\end{itemize}
