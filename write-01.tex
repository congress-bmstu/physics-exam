\subsection{Спонтанное и вынужденное излучение. Вывести соотношение Эйнштейна для
коэффициентов A и B}\label{einstein-a-b}

Для некоторой частицы с дискретным спектром энергий рассматриваем поведение при налетающем
(индуцирующим, в том плане, что оно вызывает вынужденное излучение) фотоне. Эта частица может
перейти на уровень выше, поглотив фотон, может перейти на уровень выше испустив ещё один фотон.
В случае испускания дополнительного фотона данное явление называется вынужденным излучением.

Также частица может самопроизвольно выпустить фотон, потеряв часть энергии и опустившись на
уровень ниже. Это явление называется спонтанным излучением.

Обозначим $E_1, E_2$ -- энергии основного и возбужденного состояний, $n_1, n_2$ -- количество
частиц в основном и возбужденном состоянии. Тогда за некоторый промежуток $\Delta t$ произойдёт
какое-то количество переходов $\Delta n_{12}$ из состояния 1 в 2 и $\Delta n_{21}$ из
состояния 2 в 1.

Согласно распределению Ф-Д:
\[
  N_1 \sim e^{-\dfrac{E_1}{kT}}, N_2 \sim e^{-\dfrac{E_2}{kT}}.
\]

Вспомним также такую характеристику, как спектральная плотность излучения:
\[
  u_\omega = \dfrac{dW}{dV d\omega}
\]
% TODO написать откуда следующее
тогда:
\[
  \begin{cases}
    z_{12} = \dfrac{\Delta n_{12}}{\Delta t} \sim N_1 \cdot u_\omega, \\[10pt]
    z_{21} = \dfrac{\Delta n_{21}}{\Delta t} \sim N_2. 
  \end{cases}
  \Rightarrow
  \begin{cases}
    z_{12} = B_{12} N_1 u_\omega, \\
    z_{21} = A_{21} N_2.
  \end{cases}
\]
в термодинамическом равновесии за единицу времени в оqбе стороны переходит одинаковое количество 
частиц:
\[
  z_{12} = z_{21} \Rightarrow
  B_{12} N_1 u_\omega = A_{21} N_2
  \Rightarrow
  B_{12} u_\omega = A_{21} \dfrac{N_2}{N_1}.
\]
% TODO разобраться зачем тут T->oo
Но при $T \to \infty$:
\[
  \dfrac{N_2}{N_1} = e^{- \dfrac{E_2-R_1}{kT}} \to 1,
  u_\omega = \dfrac{\hbar \omega^3}{\pi^2 c^3} \cdot \dfrac{1}{e^{\dfrac{\hbar \omega}{kT}} - 1} \to \infty.
\]
(здесь использован закон Планка).

В случае вынужденного излучения к $z_{21}$ добавяться ещё и частицы, которые выпустят фотон вследствие вынуждения:
\[
  \begin{cases}
    z_{12} = B_{12} N_1 u_\omega, \\
    z_{21} = A_{21} N_2 + B_{21} N_2 u_\omega.
  \end{cases}
\]
в термодинамическом равновесии за единицу времени в обе стороны переходит одинаковое количество 
частиц:
\[
  z_{12} = z_{21} \Rightarrow
  B_{12} N_1 u_\omega = A_{21} N_2 + B_{21} N_2 u_\omega
  \Rightarrow
  B_{12} \dfrac{N_1}{N_2} = \dfrac{A_{21}}{u_\omega} + B_{21}.
\]
Но при $T \to \infty$:
\[
  \dfrac{N_1}{N_2} = e^{- \dfrac{E_2-R_1}{kT}} \to 1,
  \dfrac{1}{u_\omega} = \dfrac{\pi^2 c^3}{\hbar \omega^3} \left(e^{\dfrac{\hbar \omega}{kT}} - 1\right) \to 0
  \Rightarrow
  \begin{cases}
    B_{12} = B_{21} = B, \\
    A_{21} = A 
  \end{cases}
\]
Таким образом:
\begin{multline*}
  B N_1 u_\omega = A N_2 + B N_2 u_\omega
  \Rightarrow
  B u_\omega \left( \dfrac{N_1-N_2}{N_2} \right) = A
  \Leftrightarrow \\
  \Leftrightarrow
  B u_\omega \left( \dfrac{N_1}{N_2} - 1 \right) = B u_\omega \left( e^{\dfrac{E_2-R_1}{kT}} - 1 \right) = A
  \Leftrightarrow
  B \dfrac{\hbar \omega^2}{\pi^2 c^3} = A
\end{multline*}
Последнее называется соотношением Эйнштейна.
