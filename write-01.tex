\subsection{Спонтанное и вынужденное излучение. Вывести соотношение Эйнштейна для
коэффициентов A и B}\label{einstein-a-b}
%ref: M, 300
Пусть в полости с веществом, состоящем из одинаковых не взаимодействующих друг с
другом частиц, наблюдается термодинамическое равновесие. Для некоторой частицы с дискретным спектром энергий рассматриваем поведение при налетающем
(индуцирующим, в том плане, что оно вызывает вынужденное излучение) фотоне. Эта частица может
перейти на уровень выше, поглотив фотон, может перейти на уровень ниже, испустив
ещё один фотон вдобавок.
В случае испускания дополнительного фотона данное явление называется
\emph{вынужденным излучением}.

Также частица может самопроизвольно выпустить фотон, потеряв часть энергии и опустившись на
уровень ниже. Это явление называется \emph{спонтанным излучение}.

Обозначим $E_1, E_2$ --- энергии основного и возбужденного состояний\footnote{В
этой упрощённой модели, считаем, что больше уровней энергии нет.}, $N_1, N_2$ --- количество
частиц в основном и возбужденном состоянии. Тогда за некоторый промежуток $\Delta t$ произойдёт
какое-то количество переходов $\Delta n_{12}$ из состояния 1 в 2 и $\Delta n_{21}$ из
состояния 2 в 1.

Согласно распределению Ф-Д (Больцмана)
\[
  N_i \sim e^{-\dfrac{E_i}{kT}},\quad \frac{N_2}{N_1} = e^{-\dfrac{E_2 - E_1}{kT}},
\]
где $ k $ --- постоянная Больцмана.

Заметим что при данных условиях, излучение в полости монохроматическое с
частотой $ \omega $ такой, что $ E_2 - E_1 = \hbar\omega $. Для падающего излучения вспомним такую характеристику, как спектральная плотность излучения:
\[
  u_\omega = \dfrac{dW}{dV d\omega}.
\]
Она характеризует, если утрировать, количество фотонов в некотором элементе объёма.

\subsubsection{Случай отсутствия вынужденного излучения}
Предположим, что при налетающем фотоне может произойти только либо поднятие $1 \to 2$, 
либо просто как-то спонтанно произойти излучение $2 \to 1$. То есть предположим, что нет никакого
вынужденного излучения и это всё ложь.

Обозначим $z_{12}$ --- количество частиц, которые за единицу времени перейдут из 1 во
2-ое состояние, $z_{21}$ -- из 2 в 1. Ясно, что каждая из этих величин пропорциональна количеству
частиц на данном энергетическом уровне. Так же ясно, что частота переходов $1\to 2$, вызываемых
поглощением фотона, пропорциональна количеству налетаемых фотонов. Тогда:
\[
  \begin{cases}
    z_{12} = \dfrac{\Delta n_{12}}{\Delta t} \sim N_1 \cdot u_\omega, \\[10pt]
    z_{21} = \dfrac{\Delta n_{21}}{\Delta t} \sim N_2. 
  \end{cases}
  \Rightarrow
  \begin{cases}
    z_{12} = B_{12} N_1 u_\omega, \\
    z_{21} = A_{21} N_2.
  \end{cases}
\]
здесь коэффициенты $B_{12}$ и $A_{21}$ появились просто как константы
пропорциональности (вероятности переходов).
В термодинамическом равновесии за единицу времени в обе стороны переходит одинаковое количество 
частиц (принцип детального равновесия):
\begin{equation}\label{write_01:bad_einstein}
  z_{12} = z_{21} \Rightarrow
  B_{12} N_1 u_\omega = A_{21} N_2
  \Rightarrow
  B_{12} u_\omega = A_{21} \dfrac{N_2}{N_1}.
\end{equation}
Рассмотрим что будет происходить при нагреве $T \to \infty$, согласно распределению Ф-Д:
\[
  \dfrac{N_2}{N_1} = e^{- \dfrac{E_2-E_1}{kT}} \to 1,
\]
Из уравнения \eqref{write_01:bad_einstein} получаем, что:
\[
  \dfrac{A_{21}}{B_{12}}
  \approx u_\omega
  = \dfrac{\hbar \omega^3}{\pi^2 c^3} \cdot \dfrac{1}{e^{\dfrac{\hbar \omega}{kT}} - 1} \to \infty.
\]
(здесь использован закон Планка) получаем противорение -- не существует таких коэффициентов, чтобы
такая модель (без вынужденного излучения) работала.

\subsubsection{Случай вынужденного излучения}
Чтобы модель стала верной, надо добавить вынужденное излучнение. В этом случае к $z_{21}$ добавяться
ещё и частицы, которые выпустят фотон вследствие вынуждения:
\[
  \begin{cases}
    z_{12} = B_{12} N_1 u_\omega, \\
    z_{21} = A_{21} N_2 + B_{21} N_2 u_\omega.
  \end{cases}
\]
в термодинамическом равновесии за единицу времени в обе стороны переходит одинаковое количество 
частиц:
\begin{equation}\label{write_1:good_einstein}
  z_{12} = z_{21} \Rightarrow
  B_{12} N_1 u_\omega = A_{21} N_2 + B_{21} N_2 u_\omega
  \Rightarrow
  B_{12} \dfrac{N_1}{N_2} = \dfrac{A_{21}}{u_\omega} + B_{21}.
\end{equation}
При $T \to \infty$:
\[
  \dfrac{N_1}{N_2} = e^{- \dfrac{E_2-R_1}{kT}} \to 1,
  \dfrac{1}{u_\omega} = \dfrac{\pi^2 c^3}{\hbar \omega^3} \left(e^{\dfrac{\hbar \omega}{kT}} - 1\right) \to 0
  \Rightarrow
  \begin{cases}
    B_{12} = B_{21} = B, \\
    A_{21} = A 
  \end{cases}
\]
Таким образом:
\begin{multline*}
  B N_1 u_\omega = A N_2 + B N_2 u_\omega
  \Rightarrow
  B u_\omega \left( \dfrac{N_1-N_2}{N_2} \right) = A
  \Leftrightarrow \\
  \Leftrightarrow
  B u_\omega \left( \dfrac{N_1}{N_2} - 1 \right) = B u_\omega \left( e^{\dfrac{E_2-E_1}{kT}} - 1 \right) = A
  \Leftrightarrow
  B \dfrac{\hbar \omega^3}{\pi^2 c^3} = A
\end{multline*}
Последнее называется соотношением Эйнштейна.
