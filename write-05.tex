\subsection{Сформулировать, какие частицы называются фермионами, привести примеры Ферми-частиц.
Сформулировать принцип Паули (W. Pauli). Используя формулу Больцмана для энтропии, вывести
распределение Ферми-Дирака} \label{fermi-dirak}
%ref: М, 352
{\footnotesize \paragraph{\footnotesize Справка.} \textsc{Число возможных распределений.}
  Задача о возможных распределениях $ N $ \textsl{неразличимых} дробинок по $ Z $ ячейкам
  \textsl{с запретом на повторение} (фермионы) 
  сводится к задаче о вытягивании $ Z $ шаров из урны с $ N $ шарами без
  возвращения, если
  мыслить себе задачу как выбор ячейки под дробинку, а не наоборот. Отсюда для
  Ферми -- Дирака получаем формулу $ W = C_Z^N $, а для Бозе -- Эйнштейна,
  аналогично, но с возвращением, $ W = C_{Z+N-1}^N $.

  Про фермионы см. \ref{princip-nerazlichimosti}.
}

\paragraph{Принцип Паули.}
\textit{Системы электронов (фермионов) встречаются в природе только в состояниях, описываемых антисимметричными волновыми функциями. }

Отсюда следует, что два одинаковых электрона (фермиона), входящих в одну систему, не могут находиться в одинаковых состояниях (иначе при перестановке волновая функция была бы чётной)\footnote{Однако отметим, что в одинаковом состоянии может находиться любое число бозонов.}.

\paragraph{Другая формулировка.} В одном и том же атоме не может быть более одного электрона с одинаковым набором четырёх квантовых чисел $n, l , m, m_s$ (про квантовые числа также см. \ref{princip-nerazlichimosti}).

\paragraph{Вывод распределения Ферми -- Дирака.}
Рассмотрим идеальный ферми-газ, то есть систему, состоящую из $N$ невзаимодействующих фермионов.
\begin{figure}[H]
	\centering
	\includegraphics[width=0.7\linewidth]{img/write-05/yacheiki}
	\caption{Возможное распределение ферми-частиц по ячейкам (черная точка - ферми частица, белая точка - отсутсвие частицы)}
	\label{fig:yacheiki}
\end{figure}
Разбив систему на $Z$ ячеек, найдем количество распределений $N$ фермионов по $Z$ ячейкам, то есть статистический вес макросостояния системы фермионов
\begin{equation*}
	W = C_Z^N = \frac{Z!}{N!(Z-N)!}.
\end{equation*}

Рассмотрим шестимерное фазовое пространство с координатами $x,y,z,p_x,p_y,p_z$. Разобьем его с помощью изоэнергетических поверхностей
\begin{equation*}
	f(x,y,z,p_x,p_y,p_z) \equiv E_i.
\end{equation*}
на тонкие слои так, что $|E_{i+1}-E_{i}|\ll E_i$. Пусть в пределы $i$-го слоя попадает $Z_i$ ячеек объемом $(2\pi\hbar)^3$ и $N_i$ частиц. Тогда аналогично примеру с раскладыванием фермионов по ячейкам получим статистический вес каждого энергетического слоя:
\begin{equation*}
	\Omega_i = \frac{Z_i!}{N_i!(Z_i-N_i)!}.
\end{equation*}
Статистический вес всей системы равен произведению статистических весов её подсистем:
\begin{equation*}
	\Omega = \prod_i \Omega_i.
\end{equation*}

Для нахождения наиболее вероятного распределения частиц по ячейкам, необходимо найти максимум статистического веса системы при условии, что количество частиц постоянно и полная энергия постоянна (жесткая адиабатическая оболочка), то есть
\begin{equation*}
	\sum N_i=\mathrm{const},\quad \sum E_iN_i=\mathrm{const}.
\end{equation*}

Вспомним формулу Больцмана для энтропии $S=k\ln \Omega$, вместо максимума $\Omega$ будем искать максимум энтропии. По свойствам логарифмов:
\begin{equation*}
	S=k\sum_i \big[\ln Z_i! - \ln N_i ! - \ln(Z_i - N_i)!\big].
\end{equation*}

Так как $N_i, Z_i \gg 1$, то воспользуемся формулой Стирлинга ($\ln n! \approx n\ln n - n$).
\begin{equation*}
	S=k\sum_i \big[
	Z_i \ln Z_i - Z_i - N_i \ln N_i + N_i - (Z_i-N_i)\ln(Z_i-N_i) + (Z_i-N_i)
	\big].
\end{equation*}
Слагаемое $Z_i \ln Z_i$ исключим из выражения, так как при поиске экстремума энтропии варьироваться будут только числа частиц в слое $N_i$, а данное слагаемое от них не зависит.

\textit{Метод множителей Лагранжа}. Рассмотрим функцию
\begin{equation*}
	F = S + \lambda_1 N + \lambda_2 E = -k\sum_i \big[N_i\ln N_i+(Z_i-N_i)\ln
  (Z_i-N_i)\big] + \lambda_1 \sum_i N_i + \lambda_2 \sum_i N_iE_i.
\end{equation*}
Приравнивая к нулю частные производные $F$ по $N_i$, получаем:
\begin{equation*}
	\frac{\partial F}{\partial N_i} = -k \left[
	\ln N_i + N_i\frac{1}{N_i}-\ln(Z_i-N_i)-(Z_i-N_i)\frac{1}{Z_i-N_i}
	\right] + \lambda_1 + \lambda_2 E_i = k \ln \frac{Z_i-N_i}{N_i} + \lambda_1 +
  \lambda_2E_i=0.
\end{equation*}
Отсюда следует, что:
\begin{equation*}
	\ln \frac{Z_i-N_i}{N_i} = -\frac{\lambda_2 E_i+\lambda_1}{k} \Leftrightarrow
	\frac{Z_i-N_i}{N_i} = \frac{1-N_i/Z_i}{N_i/Z_i}=\exp \left\{-\frac{\lambda_2
  E_i+\lambda_1}{k}\right\}.
\end{equation*}
Отношение $\frac{N_i}{Z_i}$ представляет собой среднее число фермионов $\langle
n_i \rangle$ в одной ячейке (т.е. в одном квантовом состоянии). Наиболее
вероятным значением $\langle n_i\rangle$, как следует из поиска экстремума, является:
\begin{equation*}
  \langle n_i\rangle =\frac{1}{\exp \left\{-\frac{\lambda_2
  E_i+\lambda_1}{k}\right\}+1}.
\end{equation*}

Поскольку все частные производные $F$ по $N_i$ равны нулю, то равен нулю и дифференциал этой функции $dF$, то есть:
\begin{equation*}
	dF = dS + \lambda_1 dN + \lambda_2 dE = 0.
\end{equation*}
Но так как число частиц системы постоянно ($\sum N_i=\mathrm{const}$), то
$dN=0$, а следовательно, варьируя\footnote{Какой-то физический приём,
конечно, $ E = \mathrm{const}$.}, получим
\begin{align*}
  dS + \lambda_2 dE = 0 &\implies \lambda_2 = -\frac{dS}{dE},\\
	\begin{cases}	
	dS \overset{\mathrm{def}}{=\joinrel=} \frac{\delta Q}{T},\\
  \delta Q = dE, & \text{т.к.}\, V=\mathrm{const}.
\end{cases}&\implies \lambda_2 = -\frac{dS}{dE} = -\frac{\delta Q}{TdE} =
  -\frac{1}{T}.
\end{align*}
Здесь первое уравнение системы есть следствие определения энтропии, а второе --- первый
закон термодинамики. $ \delta Q $ называют \emph{флунктуацией}.

Множитель $\lambda_1$ представим в виде $\lambda_1=\mu/T$, где $\mu$ ---
некоторая функция параметров состояния системы, в частности температуры. Можно
сказать, что таким образом мы определили $ \mu $, которую
называют \emph{химическим потенциалом}.

С учетом выражений для $\lambda_1,\,\lambda_2$, освобождаясь от индекса $i$, получаем
\begin{equation*}
  \langle n\rangle_{\textsc{Ф-Д}}=\frac{1}{\exp
  \left\{\frac{E-\mu}{kT}\right\}+1}.
\end{equation*}
распределение Ферми-Дирака. Оно определяет среднее количество фермионов, находящихся в квантовом состоянии с энергией $E$ при температуре $T$.

\begin{wrapfigure}{r}{0.5\textwidth}
	\centering
	\includegraphics[width=.8\linewidth]{img/write-05/fermi-dirak-T-notzero}
	\caption{Распределение Ферми-Дирака при $T\neq 0$}
	\label{fig:fermi-dirak-t-notzero}
\end{wrapfigure}

Химический потенциал $\mu$, очевидно, имеет размерность энергии, а в случае фермионов называется \textit{энергией Ферми} или \textit{уровнем Ферми}, обозначается $E_F$.

Анализируя выражение $\langle n\rangle_{\textsc{Ф-Д}}$ получим для $ T=0 $
\begin{align}
  \langle n\rangle_{\textsc{Ф-Д}} = \begin{cases}
    1, &E<E_F(0),\\
    0, &E>E_F(0),
	\end{cases}
\end{align}
то есть при $T=0$ распределение принимает вид ступеньки единичной высоты,
обрывающейся при $E=E_F(0)$. Для $T\neq 0$ см. рис \ref{fig:fermi-dirak-t-notzero}.
