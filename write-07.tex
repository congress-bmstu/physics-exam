\subsection{Описать явление термоэлектронной эмиссии. Рассматривая металл как потенциальный
ящик конечной глубины, заполненный «свободными» электронами, вывести формулу для плотности
тока насыщения $J_s$ термоэлектронной эмиссии.}
\label{sec:w7}
{\footnotesize \paragraph{\footnotesize Справка.} По определению \emph{дрейфовая
скорость} $ \vec v_d $ есть средняя векторная скорость всех частиц,
приобретаемая при воздействии электрического поля.

Концентрация частиц $ n $ есть их количество, делённое на объём (плотность для
тупых).
}

Термоэлектронная эмиссия -- излучение/испускание электронов из твердого тела при его нагреве.

Объектом исследования будет некий катод в вакууме, который кроме всего прочего мы нагреваем,
а вокруг него на некотором расстроянии находится анод -- в простом случае просто проводящая
пластина, с которой мы снимаем ток. Если рассматривать металл как потенциальный ящик конечной
глубины, получаем, что разность высоты потенциального барьера $U_0$ и энергии
Ферми $ E_F $ называется <<работой выхода>>
металла.
Током насыщения просто называется ток на аноде\footnote{Это практическая
  реализация идеи. Теоретически, нужно просто найти количество электронов,
испускаемых металлов и пролетающих некоторое расстояние при нагревании.}.

%TODO: идея про цилиндр \dot q? сколько заряда пролетает за dt на малой площадке
%(точке?)
Плотность тока насыщения\footnote{Здесь $ qn $ --- аналог плотности заряда, а
плотность, умноженная на скорость и есть плотность потока (можно нарисовать
цилиндр).} равна $\vec{J} = q n \vec{v_d} $, где $\vec{v_d} $ -- дрейфовая скорость, 
$n$ -- концентрация зарядов, $q$ -- заряд частицы. Ясно, что электроны некоторым образом
распределены по скоростям, а в этой формуле нас интересуют только энергии
$E > U_0 = E_F + A_B$. Соотвественно этому, $d J_x(v_x) = e v_x dn_e (v_x)$, где
$ dn_e(v_x) $ --- концентрация частиц, соответствующая скоростям в интервале $
(v_x, v_x + dv_x) $; а
\[
  dn_{e, [E, E+\Delta E]} \sim
  \dfrac{\sqrt{E}}{\exp\left(\frac{E-E_F}{kT}\right) +
  1} dE
\]
(см.
распределение свободных электронов в разделе \ref{sec:free-elecs}). Для
свободных электронов $E = p^2/(2m)$,
следовательно,
\[
  \sqrt{E} \sim p, \ dE \sim p dp \Rightarrow \sqrt{E} dE \sim p^2 dp \sim 4\pi p^2 dp = d \Omega_p,
\]
%TODO: разобраться с формулой объёма
где $d\Omega_p$ -- объём в фазовом пространстве $\vec{p} (p_x, p_y, p_z)$.
Тогда\footnote{Плотность при переходе к импульсу меняет только коэффициент.}
\[
  dn_{e, [p, p+\Delta p]} \sim \dfrac{d\Omega_p}{\exp\left(\frac{E(p) -
  E_F}{kT}\right) + 1}
\Rightarrow
dn_e (\vec{p}) = \dfrac{dp_x dp_y dp_z}{\exp\left(\frac{E(p) - E_F}{kT}\right) +
1}.
\]
Так как $v \sim p$, $\vec{v}$ сонаправлен с $\vec{p}$, то и $dn_e (v_x) \sim dn_e(p_x)$.
Учитывая также, что $E > U_0 = E_F + A_B, A_B \gg kT$, 
%TODO: почему??????
можно получить, что 
\[
  \exp\left(\dfrac{E-E_F}{kT}\right) \gg 1
  \Rightarrow
  dn_e (\vec{p}) \sim \exp\left(\dfrac{E_F}{kT}\right) \cdot \exp\left(\dfrac{-
  E(\vec{p})}{kT}\right) dp_x dp_y dp_z
\]
Дальше что-то очень простое:
\[
  E(\vec{p}) = E(p) = \dfrac{p_x^2 + p_y^2 + p_z^2}{2m_e}.
\]
Из этого найдём распределение $dn_e (p_x)$:
%TODO: расписать интегралы
\[
  dn_e (p_x) = \int_{\mathbb R^2} dn_e (\vec{p}) = e^{\dfrac{E_F}{kT}} \cdot 
  e^{- \dfrac{p_x^2}{2m_e kT}} dp_x \cdot 
  \int_\mathbb{R} e^{- \dfrac{p_y^2}{2m_e kT}} dp_y \cdot 
  \int_\mathbb{R} e^{- \dfrac{p_z^2}{2m_e kT}} dp_z
  \sim T e^{\dfrac{E_F}{kT}} \cdot e^{-\dfrac{p_x^2}{2m_e kT}} dp_x
\]
Объединяя все полученные результаты:
\begin{multline*}
  dJ_x \sim v_x \cdot dn_e (v_x) \sim p_x dn_e (p_x)
  \Rightarrow
  dJ_x \sim p_x T e^{\dfrac{E_F}{kT}} \cdot e^{-\dfrac{p_x^2}{2m_e kT}} dp_x
  \Rightarrow \\
  \Rightarrow
  J_x \sim T e^{\dfrac{E_F}{kT}} \int_{p_x, \min}^{+\infty} e^{-\dfrac{p_x^2}{2m_e kT}} p_x dp_x
  \sim T^2 e^{\dfrac{E_F}{kT}} \cdot e^{-\dfrac{(E_F + A_B)}{kT}}
  = T^2 e^{- \dfrac{A_B}{kT}}.
\end{multline*}
Здесь не очевидно было что такое $p_{x, \min}$, а это такой импульс, что энергия такой частицы
равна $U_0$, то есть $\dfrac{p_{x, \min}^2}{2m_e} = U_0$ (иные электроны просто
не вылетят).
