\subsection{Описать явление термоэлектронной эмиссии. Рассматривая металл как потенциальный
ящик конечной глубины, заполненный «свободными» электронами, вывести формулу для плотности
тока насыщения $J_s$ термоэлектронной эмиссии.}

Термоэлектронная эмиссия -- излучение/испускание электронов из твердого тела при его нагреве.

Объектом исследования будет некий катод в вакууме, который кроме всего прочего мы нагреваем,
а вокруг него на некотором расстроянии находится анод -- в простом случае просто проводящая
пластина, с которой мы снимаем ток. Если рассматривать металл как потенциальный ящик конечной
глубины, получаем, что высота потенциального барьера $U_0$ вляется т.н. <<работой выхода>>
металла.

Током насыщения просто называется ток на аноде (на самом деле в более общем случае между 
катодом и анодом ещё приложено некоторое напряжение, и в том случае током насыщения называется
именно максимальное значение тока при разных напряжениях).

Плотность тока насыщения: $\vec{J} = q n \vec{v}$, где $\vec{v}$ -- дрейфовая скорость, 
$n$ -- концентрация зарядов, $q$ -- заряд частицы. Ясно, что электроны некоторым образом
распределены по скоростям, а в этой формуле нас интересуют только энергии
$E > U_0 = E_F + A_B$. Соотвественно этому, $d J_x = e v_x dn_e (v_x)$, а
$dn_{e, [E, E+\Delta E]} \sim \dfrac{\sqrt{E}}{e^{\dfrac{E-E_F}{kT}} + 1} dE$ -- см.
распределение свободных электронов. Для свободных электронов: $E = \dfrac{p^2}{2m}$,
следовательно,
\[
  \sqrt{E} \sim p, dE \sim p dp \Rightarrow \sqrt{E} dE \sim p^2 dp \sim 4\pi p^2 p^2 dp = d \Omega_p,
\]
где $d\Omega_p$ -- объём в фазовом пространстве $\vec{p} (p_x, p_y, p_z)$. Тогда
\[
  dn_{e, [p, p+\Delta p]} \sim \dfrac{d\Omega_p}{e^\dfrac{E(p) - E_F}{kT} + 1}
  \Rightarrow
  dn_e (\vec{p}) = \dfrac{dp_x dp_y dp_z}{e^\dfrac{E(p) - E_F}{kT} + 1}
\]
так как $v \sim p$, $\vec{v}$ сонаправлен с $\vec{p}$ и $dn_e (v_x) \sim dn_e(p_x)$.
Учитывая также, что $E > U_0 = E_F + A_B, A_B \gg kT$, можно получить, что 
\[
  e^\dfrac{E-E_F}{kT} \gg 1
  \Rightarrow
  dn_e (\vec{p}) \sim e^\dfrac{E_F}{kT} \cdot e^\dfrac{- E(\vec{p})}{kT} dp_x dp_y dp_z
\]
Дальше что-то очень простое:
\[
  E(\vec{p}) = E(p) = \dfrac{p_x^2 + p_y^2 + p_z^2}{2m_e}.
\]
Из этого найдём распределение $dn_e (p_x)$:
\[
  dn_e (p_x) = \int_\Omega dn_e (\vec{p}) = e^{\dfrac{E_F}{kT}} \cdot 
  e^{- \dfrac{p_x^2}{2m_e kT}} dp_x \cdot 
  \int_\mathbb{R} e^{- \dfrac{p_y^2}{2m_e kT}} dp_y \cdot 
  \int_\mathbb{R} e^{- \dfrac{p_z^2}{2m_e kT}} dp_z
  \sim T e^{\dfrac{E_F}{kT}} \cdot e^{-\dfrac{p_x^2}{2m_e kT}} dp_x
\]
Объединяя все полученные результаты:
\[
  dJ_x \sim v_x \cdot dn_e (v_x) \sim p_x dn_e (p_x)
  \Rightarrow
  dJ_x \sim P_x T e^{\dfrac{E_F}{kT}} \cdot e^{-\dfrac{p_x^2}{2m_e kT}} dp_x
  \Rightarrow
  J_x \sim T e^{\dfrac{E_F}{kT}} \int_{p_x, \min}^{+\infty} e^{-\dfrac{p_x^2}{2m_e kT}} p_x dp_x
  \sim T^2 e^{\dfrac{E_F}{kT}} \cdot e^{-\dfrac{(E_F + A_B)}{kT}}
  = T^2 e^{- \dfrac{A_B}{kT}}
\]
здесь не очевидно было что такое $p_{x, \min}$, а это такой импульс, что энергия такой частицы
равна $U_0$, то есть $\dfrac{p_{x, \min}^2}{2m_e} = U_0$.
