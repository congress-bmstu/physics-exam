\subsection{Электропроводность $\sigma$ чистых полупроводников. Подвижность $\mu$ носителей заряда.
Эффективная масса $m^*$. Вывести формулы для концентрации $n_e$ электронов и дырок $n_h$
в чистых полупроводниках. Найти положение уровня Ферми в чистых полупроводниках при $T=0$.
Температурная зависимость собственной проводимости полупроводников.}

\paragraph{Эффективная масса} рассмотрим группу электронов полуклассически: $p = \hbar k$, тогда
групповая скорость:
\[
  v_\text{гр} = \dfrac{\partial \omega}{\partial k} = \dfrac{\partial E}{\partial p} 
  = \dfrac{\partial E}{\partial k} \left( \dfrac{\partial p}{\partial k}  \right)^{-1}
  = \dfrac{1}{\hbar} \dfrac{\partial E}{\partial k} 
\]
в полуклассическом рассмотрении: <<$\vec{F} = \dfrac{\partial \vec{p}}{\partial t},
\vec{a} = \dfrac{\partial \vec{v}}{\partial t}$>>. Тогда:
\[
  \dfrac{\partial v_\text{гр}}{\partial t}
  = \dfrac{1}{\hbar} \dfrac{\partial }{\partial t} \left( \dfrac{\partial E}{\partial k} \right)
  = \dfrac{1}{\hbar} \dfrac{\partial }{\partial k} \left( \dfrac{\partial E}{\partial k}  \right) \cdot \dfrac{\partial k}{\partial t} 
  = \dfrac{1}{\hbar^2} \dfrac{\partial^2 E}{\partial k^2} \cdot \dfrac{\partial p}{\partial t}
  = a 
\]
сопоставляя определение силы и это выражение, получаем, что эффективная масса $m^* = \hbar^2 / \dfrac{\partial^2 E}{\partial k^2}$. Такая масса учитывает, что электрон/дырка
находятся в периодическом кристалле.

\paragraph{Электропроводность твердых тел} $\vec{J} = en \vec{v}_d$, $\vec{v}_d = \vec{a} \cdot \tau$, где $\tau$ -- время <<релаксации>> между соударениями с рассеивающими элементами. Тогда:
\begin{equation}\label{write:10:v_d}
  \vec{a} = \dfrac{\vec{F}}{m^*} = \dfrac{e\vec{E}}{m^*}
  \Rightarrow
  \vec{J} = \dfrac{e^2 \tau}{m^*} n \vec{E}
  \Rightarrow
  \vec{v}_d = \dfrac{e \tau}{m^*} \vec{E} = \mu \vec{E}
\end{equation}
$\mu$ называют подвижностью. $\vec{J} = \dfrac{1}{\rho_0} \vec{E} = \sigma \vec{E}$ -- $\rho$ --
удельное сопротивление, $\sigma$ -- удельная проводимость.

\paragraph{Концентрация электронов и дырок} концентрация свободных электронов:
\[
  dn_e = f_\text{ФД} (E) dN_e = f_\text{ФД} \cdot g(E) dE, g(E) \sim \sqrt{E}.
\]
Сдвинем энергии так, чтобы 0 энергии приходился на верхнюю часть зоны валентности. 
Нас интересуют только электроны, находящиеся в зоне проводимости, то есть энергия которых
больше ширины запрещённой зоны $E_g$. Концентрацию электронов можно найти путём интегрирования:
$n_e = \int dn_e$, однако, используем также то, что $E-E_F \gg kT$, тогда в знаменателе у
распределения Ф-Д можно пренебречь единицей. Тогда:
\[
  n_e \sim \int_{E_g}^{+\infty} \sqrt{E-E_g} e^{-\dfrac{E}{kT}} dE \sim \dots
  \sim (m_e^*)^{3/2} T^{3/2} \cdot e^{\dfrac{E_F - E_g}{kT}}
\]
У дырок наоборот, рассматривается только энергии меньше нуля, для них так же верно
$E<0 \Rightarrow |E-E_F| \gg kT$, тогда
$1 - f_\text{ФД} = 1 - \dfrac{1}{e^{\dots} + 1} \approx 1 - \left( 1 - e^{\dots} \right) = e^{\dfrac{E-E_F}{kT}} $. 
\[
  n_h \sim \int_{-\infty}^0 \sqrt{-E} \cdot e^{\dfrac{E}{kT}} dE \sim (m_h^*)^{3/2} T^{3/2} e^{-\dfrac{E_f}{kT}}
\]

\paragraph{Положение уровня Ферми в чистых полупроводниках}

Когда полупроводник чистый, концентрация электронов в зоне проводимости равна концентрации электронов в зоне валентности:
\[
  n_e = n_h \Leftrightarrow (m_e^*)^{3/2} T^{3/2} e^{\dfrac{E_G - E_g}{kT}} = (m_h^*)^{3/2} T^{3/2} e^{-\dfrac{E_F}{kT}} 
  \Leftrightarrow
  E_F(T) = \dfrac{3}{4} kT \ln (\dfrac{m_h^*}{m_e^*}) + \dfrac{E_g}{2}
\]
при $T = 0$, соотвественно, уровень Ферми находится ровно по середине запрещённой зоны.

\paragraph{Температурная зависимость}
\[
  \sigma = e^2 \left( \dfrac{n_e \tau_e}{m_e^*} + \dfrac{n_h \tau_h}{m_h^*} \right) 
\]
при изменении температуры качественно меняется только время релаксации. Оценим его из классических соображений. Рассмотрим цилиндр небольшого радиуса, который окружает траекторию рассматриваемого электрона. Электрон движется с некоторой усредненной скоростью $<v>$, поэтому рассмотрим цилиндр длиной $<v> \delta t$. Количество столкновений $N$ с рассеивающими частицами пропорционально количеству частиц в этом цилиндре, а оно в свою очередь пропорционально концентрации рассеивающих частиц. Среднее время между столкновениями: $\tau = \dfrac{t}{N} \sim \dfrac{1}{n <v>}$. Рассеиваются наши электроны на фононах, которые являются бозонами, поэтому концентрация подчиняется статистике Б-Э. Поэтому:
\[
  \tau \sim \dfrac{1}{n_\text{Ф} <v>} \sim \dfrac{e^{\dfrac{E_0}{kT}} - 1}{v}
  \sim \dfrac{1}{T^{1/2} T} \sim T^{-3/2}
\]
а проводимость в таком случае:
\[
  \sigma \sim n_e \tau \sim T^{3/2} e^{-E_g / (kT)} T^{-3/2} \sim e^{-E_g / (kT)}
\]
